% supplement.tex

% specify 12pt font in the settings for your document class
\documentclass[12pt]{article}

\usepackage{october}
% \usepackage{lineno}
\begin{document}
% \setpagewiselinenumbers
% \modulolinenumbers[5]
% \linenumbers

\title{Supplement for Mathematics Texts}
\author{Stefan Lukits}
% \date{}
\maketitle
% \newcounter{expls}

\maketitle

\tableofcontents

\newpage

\section{Legendre Duality and Convex Conjugates}
\label{section:eiphaiwu}

% Here is the reason why this is useless for scoring rules. In early
% 2018, I assumed that the fact that only
% $F(x)=0.5\langle{}x,x\rangle$ is properly self-dual, ie.\
% $F(x)=F^{\ast}$ on some non-empty convex set, would give me
% $D_{H}(x,y)=D_{H}(y,x)\Rightarrow{}S(\xi_{i},x)=m(2x_{i}-1-\sum{}x_{j}^{2}$.
% The hope was that the Brier score's entropy would be self-dual in
% some sense. However, an entropy function by McCarthy's Theorem is
% always 1-homogeneous, and the formula in Jean B. Lasserre's paper
% ``Homogeneous Functions and Conjugacy'' clearly indicates that of
% homogeneous functions, ONLY 2-homogeneous functions can be
% self-dual, and ALL 1-homogeneous functions have trivial complex
% conjugates of the form $F^{\ast}=\infty$. This means that the theory
% of complex conjugates is not applicable to entropy functions for
% scoring rules. Months of work crashing down on me on July 19, 2018.

I will provide an original new proof for this theorem (there are more
conventional proofs in \scite{8}{savage71}{788}; and
\scite{7}{selten98}{}, section 4). My proof gives the reader a brief
introduction to convex conjugates, which may in many other respects be
a fruitful mathematical tool dealing with scoring rules, entropy
functions, and divergence functions. Let $\mathcal{P}$ be the set of
probability distributions over a trichotomy-type $n+1$-dimensional
outcome space. $x\in\mathcal{P}$ can be represented by the
probabilities $x_{0},{\ldots},x_{n}$ or by the vector
$x=(x_{1},{\ldots},x_{n})^{\intercal}\in\mathbb{R}^{n}$ with the usual
restrictions on probabilities. If the latter is the case, which will
be true for the rest of this subsection, then $x_{0}=1-\sum{}x_{i}$.

This differs in notation from the rest of the paper; and so does my
assumption for the rest of the subsection that $S$ is a reward (rather
than a loss) function. The reason for the inconsistency of convention
is that I will use a fair amount of convex analysis in this
subsection. The entropy functions for loss scores are concave. All
theorems of convex analysis are valid for concave functions just as
much as for convex functions, but in convex analysis, for obvious
reasons (to align convex sets with convex functions), the convention
is to use convex functions.

\begin{definition}
  \label{def:ongiepee}
  A function $f:T\rightarrow\mathbb{R}$, where $T$ is a convex subset
  of $\mathbb{R}^{n}$, is convex if its epigraph is a convex set.
\end{definition}

The epigraph $\mbox{epi}(f)$ is defined as
$\{(x,\mu)\in{}T\times\mathbb{R}\,\vert\,\mu\geq{}f(x)\}$. The more
conventional definition of convexity is that for $0<\lambda<1$,
\begin{equation}
  \label{eq:iamahquo}
  f(\lambda{}x+(1-\lambda)y)\leq\lambda{}f(x)+(1-\lambda)f(y)\mbox{ for all }x,y\in{}T.
\end{equation}
Theorem 4.1 in \scite{8}{rockafellar97}{25}, shows these two
definitions of convex functions to be equivalent.

The reward function
$S_{\mbox{\tiny LS}}:\mathbb{R}^{n}\rightarrow\mathbb{R}^{n+1}$ for the Log
score is
% \begin{equation}
%   \label{eq:quahkeev}
%   S_{\mbox{\tiny LS}}(x)=\left(\ln{}x_{i}\right)_{i=1,{\ldots},n}^{\intercal}
% \end{equation}
\begin{equation}
  \label{eq:quahkeev}
  S_{\mbox{\tiny LS}}(x)=\left(\ln\left(1-\sum{}x_{i}\right),\ln{}x_{1},{\ldots},\ln{}x_{n}\right)^{\intercal}
\end{equation}
and the entropy function is
\begin{equation}
  \label{eq:eeceetie}
  H_{\mbox{\tiny LS}}(x)=-\sum_{i=1}^{n}x_{i}\ln{}x_{i}-\left(1-\sum_{i=1}^{n}x_{i}\right)\ln\left(1-\sum_{i=1}^{n}x_{i}\right).
\end{equation}
I will use {\quations}~(\ref{eq:quahkeev}) and (\ref{eq:eeceetie}) to
illustrate convex conjugates.

Let the entropy function $H$ for the scoring rules under consideration
be a twice differentiable, strictly convex function on $\mathcal{P}$.
For the remainder of this subsection, I will assume commutativity of
the inner product and leave aside technicalities about the inner
product between dual spaces.

\begin{definition}
  \label{def:paegoozi}
  The convex conjugate
  $H^{\ast}:\mathcal{P}^{\ast}\rightarrow\mathbb{R}$ is defined on
\begin{equation}
  \label{eq:osaenise}
\mathcal{P}^{\ast}=\{x^{\ast}\in\mathbb{R}^{n}\,\vert\,\nabla{}H(x)=x^{\ast},x\in\mathcal{P}\}
\end{equation}
by
\begin{equation}
  \label{eq:veebuave}
  H^{\ast}(x^{\ast})=\sup_{x\in\mathcal{P}}\left\{\langle{}x,x^{\ast}\rangle-H(x)\right\}.
\end{equation}
\end{definition}

Based on the strict convexity and differentiability of $H$,
$\mathcal{P}$ and $\mathcal{P}^{\ast}$ are dual spaces and
\begin{equation}
  \label{eq:phavuugi}
  H^{\ast}(x^{\ast})=\langle{}x,x^{\ast}\rangle-H(x)
\end{equation}
where $x$ is the dual element corresponding to $x^{\ast}$, i.e.\
$\nabla{}H(x)=x^{\ast}$ and the supremum in (\ref{eq:veebuave}) is a
maximum, if it exists.

Convex conjugates induce the so-called Legendre-Fenchel duality
between convex differentiable functions on $\mathcal{P}$ and
$\mathcal{P}^{\ast}$. 

% It can be shown that $H^{\ast\ast}=H$. Section
% 12 of R. Tyrrell Rockafellar's book \emph{Convex Analysis} has the
% details. 

{\Emmas}~\ref{lma:kengohja}, \ref{lma:pudiubao}, and
\ref{lma:piangodi} are adapted from section 12 of R. Tyrrell
Rockafellar's book \emph{Convex Analysis}.

\begin{lemma}
  \label{lma:kengohja}
    $H^{\ast\ast}=H$.
\end{lemma}

\begin{proof}
  \label{prf:pivailah}
  First, I will show that $H^{\ast}$ is convex, otherwise duality
  cannot be maintained because $x$ would no longer have a unique dual
  element $x^{\ast}$. The definition of $H^{\ast}$ immediately
  suggests that
  \begin{equation}
    \label{eq:phuuquoo}
    H^{\ast}(x^{\ast})=\sup_{(x,\mu)\in\mbox{\scriptsize epi}(H)}\{\langle{}x,x^{\ast}\rangle-\mu\}
  \end{equation}
  Therefore,
  \begin{equation}
    \label{eq:reicedik}
    \mbox{epi}(H^{\ast})=\{(x^{\ast},\nu)\,\vert\,\nu\geq\sup_{(x,\mu)\in\mbox{\scriptsize epi}(H)}\{\langle{}x,x^{\ast}\rangle-\mu\}
  \end{equation}
Let $(x_{1}^{\ast},\nu_{1})$ and $(x_{2}^{\ast},\nu_{2})$ be elements
of $\mbox{epi}(H^{\ast})$. Then
$(\lambda{}x_{1}^{\ast}+(1-\lambda)x_{2}^{\ast},\lambda\nu_{1}+(1-\lambda)\nu_{2})$
is also an element of $\mbox{epi}(H^{\ast})$ because for fixed
$x_{1}^{\ast},x_{2}^{\ast},\nu_{1},\nu_{2}$,and $0<\lambda<1$
\begin{equation}
  \label{eq:saedeeva}
  \begin{split}
  & \lambda\nu_{1}+(1-\lambda)\nu_{2}\geq\lambda\sup_{(x,\mu)\in\mbox{\scriptsize epi}(H)}\{\langle{}x,x_{1}^{\ast}\rangle-\mu\}+(1-\lambda)\sup_{(x,\mu)\in\mbox{\scriptsize epi}(H)}\{\langle{}x,x_{2}^{\ast}\rangle-\mu\}\geq \\
  & \sup_{(x,\mu)\in\mbox{\scriptsize epi}(H)}\{\lambda(\langle{}x,x_{1}^{\ast}\rangle-\mu)+(1-\lambda)(\langle{}x,x_{2}^{\ast}\rangle-\mu)\}= \\
  & \sup_{(x,\mu)\in\mbox{\scriptsize epi}(H)}\{\langle{}x,\lambda{}x_{1}^{\ast}+(1-\lambda)x_{2}^{\ast}\rangle-\mu\}
  \end{split}
\end{equation}
This establishes convexity for $H^{\ast}$. Consider
\begin{equation}
  \label{eq:deluuqua}
  \sup_{(x^{\ast},\nu)\in\mbox{\scriptsize epi}(H^{\ast})}\{\langle{}x,x^{\ast}\rangle-\nu\}
\end{equation}
On the one hand, (\ref{eq:deluuqua}) is $H^{\ast\ast}(x)$ by
definition. On the other hand, it is
\begin{equation}
  \label{eq:geifahve}
  \sup_{x^{\ast}\in\mathcal{P}^{\ast}}\{\langle{}x,x^{\ast}\rangle-H^{\ast}(x^{\ast})\}=\sup_{x^{\ast}\in\mathcal{P}^{\ast}}\left\{\langle{}x,x^{\ast}\rangle-\sup_{\xi\in\mathcal{P}}\{\langle\xi,x^{\ast}\rangle-H(\xi)\}\right\}
\end{equation}
The inside supremum is achieved where $\nabla{}H(\xi)=x^{\ast}$
because $H$ is differentiable. Differentiate
$\langle{}x-\xi,\nabla{}H(\xi)\rangle+H(\xi)$, using the product rule
for the inner product, to see that the outside supremum is achieved
where $\nabla{}H(x)=x^{\ast}$. The convexity of $H$ demands that
$x=\xi$, and it is therefore established that (\ref{eq:deluuqua}) is
not only $H^{\ast\ast}(x)$, but also $H(x)$, so $H^{\ast\ast}=H$.
\end{proof}

\begin{lemma}
  \label{lma:pudiubao}
  Taking conjugates reverses functional inequalities:
  $f_{1}\leq{}f_{2}$ implies $f_{1}^{\ast}\geq{}f_{2}^{\ast}$. 
\end{lemma}
\begin{proof}
  \label{prf:aibaelee}
  A real-valued function $f_{2}$ is greater or equal to another
  real-valued function $f_{1}$ on a common domain if on that common
  domain $f_{1}(x)\leq{}f_{2}(x)$. Let $f_{1}\leq{}f_{2}$. Then
  \begin{equation}
    \label{eq:zeewooje}
    f_{2}^{\ast}(x^{\ast})=\sup_{x\in\mathcal{P}}\{\langle{}x,x^{\ast}\rangle-f_{2}(x)\}\leq\sup_{x\in\mathcal{P}}\{\langle{}x,x^{\ast}\rangle-f_{1}(x)\}=f_{1}^{\ast}(x^{\ast})
  \end{equation}
for all $x^{\ast}\in\mathcal{P}^{\ast}$.
\end{proof}

\begin{lemma}
  \label{lma:piangodi}
  $\langle{}x,x^{\ast}\rangle\leq{}f(x)+f^{\ast}(x^{\ast})$ for all
  $x\in\mathcal{P},x^{\ast}\in\mathcal{P}^{\ast}$ (not necessarily
  dual elements as above), and any differentiable convex function $f$
  and its conjugate $f^{\ast}$.
\end{lemma}
\begin{proof}
  \label{prf:mohgeeph}
Consider for fixed $x,x^{\ast}$
\begin{equation}
  \label{eq:poyishoo}
  f^{\ast}(x^{\ast})=\sup_{\xi\in\mathcal{P}}\{\langle\xi,x^{\ast}\rangle-f(\xi)\}\geq\langle{}x,x^{\ast}\rangle-f(x)
\end{equation}
$\langle{}x,x^{\ast}\rangle\leq{}f(x)+f^{\ast}(x^{\ast})$ follows.
\end{proof}

% The proofs for {\emmas}~\ref{lma:pudiubao} and \ref{lma:piangodi} are in
% \scite{8}{rockafellar97}{104f}.

The inequality in {\emma}~\ref{lma:piangodi} is called Fenchel's
inequality.

\begin{lemma}
  \label{lma:xeemeixo}
  $\left(\nabla{}H\right)^{-1}=\nabla\left(H^{\ast}\right)$.
\end{lemma}
\begin{proof}
  \label{prf:ezegheop}
$H$ is a differentiable and strictly convex function on $\mathcal{P}$,
therefore $\nabla{}H$ exists and is bijective from $\mathcal{P}$ to
$\mathcal{P}^{\ast}$. Let $(\nabla{}H)^{-1}$ be the inverse function
of $\nabla{}H$. According to (\ref{eq:phavuugi}), the supremum in
(\ref{eq:veebuave}) is reached at $\nabla{}H(x)=x^{\ast}$, therefore
\begin{equation}
  \label{eq:ohcaezah}
  H^{\ast}(\nabla{}H(x))=\langle{}x,\nabla{}H(x)\rangle-H(x)
\end{equation}
Differentiating (\ref{eq:ohcaezah}) yields
\begin{equation}
  \label{eq:uephaele}
\nabla{}H^{\ast}(\nabla{}H(x))\cdot\nabla^{2}H(x)=\nabla{}H(x)+x\cdot\nabla^{2}H(x)-\nabla{}H(x)
\end{equation}
This can only be true if $\nabla{}H^{\ast}\circ\nabla{}H=$id, which
establishes {\emma}~\ref{lma:xeemeixo}. My proof is based on
\scite{8}{boissonnatetal10}{287}. Rockafellar treats the topic with greater
generality, where $H$ need not be differentiable, see his Theorem 23.5
on page 218 and his Theorem 26.5 on page 258.
\end{proof}

% Let $D_{H}$ be the Bregman divergence associated with $H$ defined by
% (see {\efinition}~\ref{def:logahpoh})
% \begin{equation}
%   \label{eq:thiepooz}
% D_{H}(x\|y)=H(x)-H(y)-\langle{}x-y,\nabla{}H(y)\rangle.
% \end{equation}

To keep the Bregman divergence corresponding to the entropy function
positive, it will in this subsection also be adjusted to rewards
rather than losses and defined in accordance with
\texttt{def:logahpoh}
\begin{equation}
  \label{eq:pooceiyu}
  D_{H}(x\|y)=H(x)-H(y)+\langle{}y-x,\nabla{}H(y)\rangle.
\end{equation}

\begin{lemma}
  \label{lma:ohviomov}
$D_{H}(x\|y)=D_{H^{\ast}}\left(y^{\ast}\|x^{\ast}\right)$.
\end{lemma}
\begin{proof}
  \label{prf:keeseiph}
  Recall that $H(x)+H^{\ast}(x^{\ast})=\langle{}x,x^{\ast}\rangle$ for
  $\nabla{}H(x)=x^{\ast}$ according to (\ref{eq:phavuugi}). Together
  with {\emma}~\ref{lma:xeemeixo}, the bilinearity and commutativity of
  the inner product, and \texttt{eq:thiepooz} this yields
\begin{equation}
  \label{eq:chahgheo}
  \begin{split}
& D_{H}(x\|y)-D_{H^{\ast}}\left(y^{\ast}\|x^{\ast}\right)= \\
& H(x)-H(y)-\langle{}x-y,\nabla{}H(y)\rangle-H^{\ast}(y^{\ast})+H^{\ast}(x^{\ast})+\langle{}y^{\ast}-x^{\ast},\nabla{}H^{\ast}(x^{\ast})\rangle= \\
& H(x)-H(y)-\langle{}x,\nabla{}H(y)\rangle+\langle{}y,y^{\ast}\rangle-H^{\ast}(y^{\ast})+ \\
& H^{\ast}(x^{\ast})+\langle{}\nabla{}H(y),\nabla{}H^{\ast}(\nabla{}H(x))\rangle-\langle{}x^{\ast},\nabla{}H^{\ast}(\nabla{}H(x))\rangle= \\
& H(x)-H(y)+\langle{}y,y^{\ast}\rangle-H^{\ast}(y^{\ast})+H^{\ast}(x^{\ast})-\langle{}x^{\ast},x\rangle=0 \\
  \end{split}
\end{equation}
and establishes the lemma. The proof is based on \scite{8}{boissonnatetal10}{287}.
\end{proof}

My illustration for convex conjugates is the Log score (the Brier
score would make a poor illustration, because, as we will find out,
the Brier score is uniquely self-dual). For a given $x\in\mathcal{P}$,
differentiate (\ref{eq:eeceetie}) to define $x^{\ast}$
\begin{equation}
  \label{eq:eipaiwup}
  x^{\ast}_{i}=\frac{\partial}{\partial{}x_{i}}H_{\mbox{\tiny LS}}(x)=\ln\left(1-\sum_{j=1}^{n}x_{j}\right)-\ln{}x_{i}.
\end{equation}
To find $\nabla{}H_{\mbox{\tiny LS}}^{\ast}$, which is equal to
$\left(\nabla{}H_{\mbox{\tiny LS}}\right)^{-1}$ according to {\emma}~\ref{lma:xeemeixo},
solve the system of equations
\begin{equation}
  \label{eq:jahngohr}
  \sum_{j=1}^{n}\left(e^{x^{\ast}_{i}}+\delta_{ij}\right)x_{j}=1.
\end{equation}
I found this solution by playing around with traces, inverses, and
determinants on octave. Schmierbuch page 2510 has a summary of the
conjecture I made on the basis of this playing around. You'd have to
do some linear algebra to find a proof for this conjecture.
\begin{equation}
  \label{eq:yahxiexo}
  x_{i}=\frac{1+\left(\sum_{i=1}^{n}e^{x^{\ast}_{i}}\right)-ne^{x^{\ast}_{i}}}{1+\left(\sum_{i=1}^{n}e^{x^{\ast}_{i}}\right)}.
\end{equation}
Because of {\emma}~\ref{lma:xeemeixo}, just as
$\nabla{}H_{\mbox{\tiny LS}}(x)$ corresponds to the right-hand side of
(\ref{eq:eipaiwup}), $\nabla{}H_{\mbox{\tiny LS}}^{\ast}(x^{\ast})$
corresponds to the right-hand side of (\ref{eq:yahxiexo}).
$H_{\mbox{\tiny LS}}^{\ast}$ is an antiderivative of the partial
derivatives in (\ref{eq:yahxiexo}) with the constant of integration
fixed by {\efinition}~\ref{def:paegoozi}.

% For multiple dimensions, the linear algebra soon turns into a chore,
% but for $n=1$,
% \begin{equation}
%   \label{eq:aehahkae}
%   x=\frac{1}{e^{x^{\ast}}+1},x^{\ast}\in\mathbb{R}=\mathcal{P}^{\ast}
% \end{equation}
% with $H^{\ast}$ an antiderivative of the expression in
% (\ref{eq:aehahkae}) (the constant of integration follows from applying
% {\efinition}~\ref{def:paegoozi}),
% \begin{equation}
%   \label{eq:ueheisha}
%   H^{\ast}(x^{\ast})=x^{\ast}-\ln\left(e^{x^{\ast}}+1\right).
% \end{equation}
% For $n=2$,
% \begin{equation}
%   \label{eq:raeyevua}
  % x_{1}=\frac{\left(e^{x^{\ast}_{2}}+1\right)e^{x^{\ast}_{1}}-e^{x^{\ast}_{1}+x^{\ast}_{2}}}{e^{x^{\ast}_{1}}+e^{x^{\ast}_{2}}+1}\mbox{ and }x_{2}=\frac{\left(e^{x^{\ast}_{1}}+1\right)e^{x^{\ast}_{2}}-e^{x^{\ast}_{1}+x^{\ast}_{2}}}{e^{x^{\ast}_{1}}+e^{x^{\ast}_{2}}+1}
%   x_{1}=\frac{e^{x^{\ast}_{2}}+1-e^{x^{\ast}_{1}}}{e^{x^{\ast}_{1}}+e^{x^{\ast}_{2}}+1}\mbox{ and }x_{2}=\frac{e^{x^{\ast}_{1}}+1-e^{x^{\ast}_{2}}}{e^{x^{\ast}_{1}}+e^{x^{\ast}_{2}}+1}
% \end{equation}
% with $H^{\ast}$ an antiderivative of the partial derivatives in
% (\ref{eq:raeyevua}). The reader can by now probably appreciate how
% convoluted it is to find convex conjugates in general except for
% certain functions that lend themselves to this procedure.
% there is an explicit formula here:
% https://www.sonycsl.co.jp/person/nielsen/Note-LegendreTransformation.pdf
% equation 4

\begin{lemma}
  \label{lma:opheejai}
  The only self-dual convex differentiable function on $\mathcal{P}$
  is $F(x)=\frac{1}{2}\langle{}x,x\rangle$.
\end{lemma}
\begin{proof}
  \label{prf:vaegheiv}
  I am adapting a proof in \scite{8}{rockafellar97}{106}. That $F$ is
  self-dual follows from {\efinition}~\ref{def:paegoozi} and
  {\emma}~\ref{lma:xeemeixo}. Now let $G$ be a self-dual convex
  differentiable function. By {\emma}~\ref{lma:piangodi} (Fenchel's
  inequality),
\begin{equation}
  \label{eq:oogoohei}
  \langle{}x,x\rangle\leq{}G(x)+G^{\ast}(x)=2G(x).
\end{equation}
This means that $G\geq{}F$. By {\emma}~\ref{lma:pudiubao},
$G^{\ast}\leq{}F^{\ast}$. Together with the self-duality of both $F$
and $G$, this implies $F=G$.
\end{proof}

% {\Emmas}~\ref{lma:xeemeixo}, \ref{lma:ohviomov}, and
% \ref{lma:opheejai} establish {\heorem}~\ref{thm:iechohng}.

I will use the preceding lemmas to prove \texttt{thm:iechohng}
stated in the beginning of the subsection, showing that only the Brier
score and its close relatives are symmetric.

\begin{proof}
  \label{prf:quojeebo}
  Recall that in the context of convex
  functions, I am using reward functions rather than loss functions.
  With this adjustment, the reward function and entropy function for
  the Brier score according to (\ref{eq:hiexaeji}) and
  (\ref{eq:yeuthohn}) are
\begin{equation}
  \label{eq:lethaiqu}
    S(\xi_{i},x)=\frac{2x_{i}}{\sum_{k}x_{k}}-1-\sum_{j}\left(\frac{x_{j}}{\sum_{k}x_{k}}\right)^{2}
\end{equation}
\begin{equation}
  \label{eq:iviniewe}
    H(x)=\sum_{i}{}x_{i}\left(\frac{2x_{i}}{\sum_{k}x_{k}}-1-\sum_{j}\left(\frac{x_{j}}{\sum_{k}x_{k}}\right)^2\right).
\end{equation}
Since I assume probabilism, the sum of the probabilities is one. The
above equations simplify to
\begin{equation}
  \label{eq:doteeboo}
    S(\xi_{i},x)=2x_{i}-1-\sum_{j}x_{j}^{2}
\end{equation}
\begin{equation}
  \label{eq:piihieri}
    H(x)=\sum_{i}{}x_{i}\left(2x_{i}-1-\sum_{j}x_{j}^{2}\right).
\end{equation}

Note that according to McCarthy's theorem
\texttt{thm:lahpoodu} the $i$-th component of the vector
$\nabla{}H(x)$ equals $S(\xi_{i},x)$ and therefore (see \texttt{cor:quiphaef})
\begin{equation}
  \label{eq:idiirioj}
  H(x)=\sum_{i=1}^{n}x_{i}S(\xi_{i},x)
\end{equation}
For a close relative of the Brier score, therefore, reward function
and entropy function are
\begin{equation}
  \label{eq:paegeina}
    S(\xi_{i},x)=m\left(2x_{i}-1-\sum_{j}x_{j}^{2}\right)+b
\end{equation}
\begin{equation}
  \label{eq:thughica}
    H(x)=m\left(\sum_{i}{}x_{i}\left(2x_{i}-1-\sum_{j}x_{j}^{2}\right)\right)+b
\end{equation}
for some $m\in\mathbb{R}^{+},b\in\mathbb{R}$. The following
calculation shows that the Brier score and its close relatives are
symmetric.
\begin{equation}
  \label{eq:uudeilah}
  \begin{split}
    & D(x\|y)-D(y\|x)=H(x)-H(y)+\langle{}y,\nabla{}H(y)\rangle-H(y)+H(x)-\langle{}x,\nabla{}H(x)\rangle= \\
    & 2\left(m\left(\sum_{i}{}x_{i}\left(2x_{i}-1-\sum_{j}x_{j}^{2}\right)\right)+b\right)- \\
    & \sum_{i}x_{i}\left(m\left(2x_{i}-1-\sum_{j}x_{j}^{2}\right)+b\right)-\sum_{i}x_{i}\left(m\left(2y_{i}-1-\sum_{j}y_{j}^{2}\right)+b\right)- \\
    & 2\left(m\left(\sum_{i}{}y_{i}\left(2y_{i}-1-\sum_{j}y_{j}^{2}\right)\right)+b\right)+ \\
    & \sum_{i}y_{i}\left(m\left(2y_{i}-1-\sum_{j}y_{j}^{2}\right)+b\right)+\sum_{i}y_{i}\left(m\left(2x_{i}-1-\sum_{j}x_{j}^{2}\right)+b\right)=0
  \end{split}
\end{equation}
\end{proof}

\section{Hector and Paris}
\label{section:shauboda}

\subsection{Setup}
\label{subsection:iewohgae}

Hector and Paris have one dollar to invest in bets on a Bernoulli coin
toss. A Bernoulli coin toss models whether or not an event will take
place. Am I HIV-positive or not? Will I inherit a large sum of money
in the next three months? Will this physical coin toss land heads or
tails? Is the next die roll going to be a five? Will this stock
increase by a certain amount in the next two weeks? Betting on a
Bernoulli coin toss is related to the partial belief that an agent has
about whether the coin toss lands heads. Investing in the stock market
or shorting a stock is in some ways like betting on a Bernoulli coin
toss; so is paying a certain price for a certain good. 

Hector and Paris's betting strategy is represented by a function
$R:[0,1]\rightarrow[0,1]$. If $w$ represents the bets offered to them,
$R(w)$ is their response. The offer works as follows. Hector and Paris
can invest in an $H$-bet at $w$ or a $T$-bet at $1-w$. I will call
their investments the investment schema; I will call the two bets offered
to them ($H$-bet at $w$ and $T$-bet at $1-w$) the bets associated with
$w$.

Let us say one of them invests $v_{H}$ dollars in an $H$-bet at $w$.
Then his prize money is zero dollars if $T$ is tossed; $v_{H}/w$ if $H$ is
tossed. The gain is $-v_{H}$ for $T$; $(v_{H}-v_{H}w)/w$ for $H$. The
investment of $v_{T}$ in a $T$-bet at $1-w$ works likewise.
Consequently, for investment schema
\begin{equation}
  \label{eq:fogohthe}
(v_{H},v_{T})\in\{(v_{H},v_{T})\in\mathbb{R}^{2}|0\leq{}v_{H},0\leq{}v_{T},v_{H}+v_{T}\leq{}1\}
\end{equation}
the gain is
\begin{equation}
  \label{eq:githoofa}
  \frac{v_{H}-v_{T}w-v_{H}w}{w}\hspace{1in}\mbox{ if }H\mbox{ is tossed}
\end{equation}

and

\begin{equation}
  \label{eq:eishuosa}
  \frac{-v_{H}+v_{T}w+v_{H}w}{1-w}\hspace{1in}\mbox{ if }T\mbox{ is tossed}
\end{equation}

\begin{proposition}
  \label{prp:ahjahfan}
  For any investment schema $(v_{H},v_{T})$ there is an investment
  schema $(v_{H}',v_{T}')$ with the same outcome as $(v_{H},v_{T})$ and
  $v_{H}'+v_{T}'=1$.
\end{proposition}

\begin{proof}
  \label{prf:xaethaef}
  Define $v_{H}'=v_{H}+w-w(v_{H}+v_{T})$ and
  $v_{T}'=1-v_{H}'$. Plugging $v_{H}'$ and $v_{T}'$ into
  {\quations}~\ref{eq:githoofa} and \ref{eq:eishuosa} leaves the
  resulting gain unchanged, but now $v_{H}'+v_{T}'=1$. 
\end{proof}

As a consequence, the investment schemas $(v_{H}',v_{T}')$ where
$v_{H}'+v_{T}'=1$ are representative for all other possible investment
schemas. We shall identify the investment schema $(v_{H}',v_{T}')$ by
$x=v_{H}'$ ($v_{T}'$ is uniquely determined by $1-x$). The function
$R$ assigns to each set of bets associated with $w$ the investment
schema $R(w)=x$.

If a betting agent wants to bail (not bet), they simply invest
$(w,1-w)$. This strategy guarantees them a gain of zero, no matter
what the result of the toss is. For a betting agent who always wants
to bail, $R(w)=w$ for all $w\in[0,1]$.

$R$ is in some way indicative of the agent's partial belief in the
outcome of the coin toss. It may be indicative of other things as
well, for example the risk tolerance of the agent or how much
information the agent believes to have about the circumstances of the
coin toss. For the purposes of this paper, the agent models the coin
toss as the outcome of a random generator. This means that at least in
the agent's model there is an objective chance $c$ that the coin toss
will land heads. The agent usually does not know this number and might
have partial beliefs about it.

% Let the probability density $P$
% represent the agent's credences about $c$, such that the probability
% that $c\leq{}x$ is
% \begin{equation}
%   \label{eq:diimorah}
%   \int_{0}^{x}dP
% \end{equation}

Let the cumulative credence function $F$ for the objective chance $c$ be
\begin{equation}
  \label{eq:vooshoov}
  \mbox{Cr}(c\leq{}y)=F(y)
\end{equation}
where Cr is the credence (in this case that the objective chance $c$
is smaller than or equal to $y$). Furthermore, assume that $F$ is
absolutely continuous so that there is a Lebesgue-integrable density
function $f$ with
\begin{equation}
  \label{eq:ovohdaiy}
  F(y)=\int_{-\infty}^{y}f(u)\,du
\end{equation}

Let $G(x,w)$ be the gain of investing $x$ in an $H$-bet at $w$ and
$1-x$ in a $T$-bet at $1-w$. There are two values for $G(x,w)$, one if
$H$ is tossed, another if $T$ is tossed.
\begin{equation}
  \label{eq:boxobohm}
  G(x,w)=\left(\frac{x-w}{w},-\frac{x-w}{1-w}\right)
\end{equation}

\begin{proposition}
  \label{prp:tohboogh}
  The expected gain for an agent holding Cr and following the
  investment schema $x$ upon being offered the set of bets associated
  with $w$ is
\begin{equation}
  \label{eq:aipheeke}
  E(x,w)=\frac{x-w}{w(1-w)}\left(\int{}yf(y)\,dy-w\right)
\end{equation}
\end{proposition}

\begin{proof}
  \label{prf:dauchahs}
  Simplify
  \begin{equation}
    \label{eq:choobich}
    E(x,w)=\int{}f(y)\left(y\left(\frac{x}{w}-1\right)+(1-y)\left(-\frac{x-w}{1-w}\right)\right)\,dy
  \end{equation}
  using $\int{}f(y)\,dy=1$.
\end{proof}
% Schmierbuch page 2540

$\int{}yf(y)\,dy$ is the agent's sharp credence in $H$. Clearly,
$E(x,w)$ is maximized with $x=1$ when the sharp credence is strictly
greater than $w$; and with $x=0$ when the sharp credence is strictly
less than $w$.

\section{Zillner Map for Revising the Geometry of Reason}
\label{section:xudahdoe}

Start off with Clinton/Trump and propriety. Restrict everything to
finite. Introduce tetrahedron. Outline Pettigrew's symmetry argument.
Cast doubt on symmetry and geometry of reason. (i) Reductio using
Levinstein et al. (ii) The oddity of Pettigrew's use of Bronfman.
Build up the case for the logarithmic measure. Summarize Shannon's and
Kullback-Leibler's arguments. Show how the logarithmic measure
generalizes Bayes and Jeffrey. But then there are oddities with the
logarithmic measure as well. Solution: divorce yourself even more from
geometry.

Make a list of desiderata, everything from the literature, in
particular
\begin{description}
\item[symmetry] see Savage for equivalence Brier score and symmetry;
  Ovcharov references Boissonnat for this fact---it would be
  interesting to see if it could be proved using Legendre duality
\item[propriety] uncontroversial
\item[entropy] have an entropy function that matches Shannon's requirements
\item[locality] make reward only depend on $p_{i}$ (Goodness, more detailed
  description in Dawid, Lauritzen, Parry, page 594; Savage, page 794)
\item[horizon] horizon
\item[reductio-resistance] especially Levinstein etc.
\item[Bronfman] Pettigrew's argument against diversity
\item[additivity] a Pettigrew argument; both BS and LS fulfill it
\end{description}

Perhaps you can make a case that Brier and Log are the last two
candidates standing. Pettigrew's case is that Brier is the only
reasonable symmetric SR, whereas there is more diversity on the
asymmetric side and therefore vulnerability to Bronfman.

Open questions:

\begin{itemize}
\item Can the Shannon entropy be extended to non-standard
  probabilities? Answer: No luck on google.

\item In what sense is the logarithmic score a procedure that leads to
  MLE (Dawid et al, 594)? For an answer to this question see Gneiting
  and Raftery, pages 372ff.
\item How can you fix things so that $D_{\mbox{\tiny KL}}$ is the
  divergence function of the logarithmic score? Answer:
  McCarthy's theorem.
\item The above procedure shows how you can move from scoring rules to
  entropy and divergence. What is the reverse procedure using
  gradients? Answer: McCarthy's theorem.
\item Does BdF's argument go through if the distance function is a
  divergence? The answer to this question is in Theorem I.D.5 in
  \texttt{richard-pettigrew-chapter-08.pdf}, page 13 (page 92 in the
  book); and the answer is yes. BdF's argument is valid for any
  additive Bregman divergence.
\item Where is the symmetry argument? Can it be rephrased in terms of
  Legendre duality? The first question appears to be answered in
  Gneiting and Raftery, page 361: the symmetry argument is original to
  Savage 1971.
\item What is the connection between the $H$ of HB and entropy? Here
  is a comment in my annotation on Gneiting and Raftery, page 361: ``$G$
  measures what I will score if $Q$ is nature's chance function. I score
  best when I predict $Q$ and the SR is proper. The entropy function
  indicates my maximum reward if I correctly guess at $Q$. The more
  uncertainty the more reward. Why would this be true? Why would I get
  a higher reward for correctly guessing a more uncertain
  distribution?''
\end{itemize}

\section{Jean-Daniel Boissonnat, Frank Nielsen, and Richard Nock:
  Bregman Voronoi Diagrams}
\label{section:enateegh}

This paper is inspirational, especially with respect to convex
conjugates. For example, it seems to me you should be able to prove
that symmetry and Brier score go together on the basis of convex
conjugates. First, let me try to find the convex conjugates for BS and
LS (Brier score and Log score). I have verified the following in SB
pages 2441ff. For LS, the loss function is
\begin{equation}
  \label{eq:oneeraiw}
  \mbox{(LS) }S(\xi_{i},x)=\ln{}x_{i}-\ln\sum{}x_{k}
\end{equation}
on the positive orthant (convex cone containing $\mathcal{P}$). For
BS, it is
\begin{equation}
  \label{eq:hiexaeji}
  \mbox{(BS) }S(\xi_{i},x)=\frac{2x_{i}}{\sum{}x_{k}}-1-\sum_{j}\left(\frac{x_{j}}{\sum{}x_{k}}\right)^{2}
\end{equation}
The corresponding entropy functions are
\begin{equation}
  \label{eq:yeuthohn}
  \mbox{(LS) }H(x)=\sum_{i}{}x_{i}\ln\frac{x_{i}}{\sum{}x_{k}}
\end{equation}
\begin{equation}
  \label{eq:ahfooyai}
  \mbox{(BS) }H(x)=\sum_{i}{}x_{i}\left(\frac{2x_{i}}{\sum{}x_{k}}-1-\sum_{j}\left(\frac{x_{j}}{\sum{}x_{k}}\right)^2\right)
\end{equation}
I have verified using sage that for both LS and BS it is true that
\begin{equation}
  \label{eq:eikughoh}
  \nabla{}H(x)=S(\xi,x)
\end{equation}
where $S(\xi,x)=(S(\xi_{1},x),{\ldots},S(\xi_{n},x))^{\intercal}$ and
\begin{equation}
  \label{eq:esoobiey}
  \nabla{}H(x)=\left(\frac{\partial{}H}{\partial{}x_{1}},{\ldots},\frac{\partial{}H}{\partial{}x_{n}}\right)^{\intercal}
\end{equation}
The scoring rule is the derivative of the entropy function. The
entropy function generates a divergence via (see Boissonnat, 241)
\begin{equation}
  \label{eq:boodoquu}
  D(p\|q)=H(p)-H(q)-\langle{}p-q,\nabla{}H(q)\rangle
\end{equation}
where $\langle{}.,.\rangle$ is the dot product (or matrix
multiplication if you are using dual spaces). In the case of LS, the
divergence is the Kullback-Leibler divergence (verified in SB 2440f),
\begin{equation}
  \label{eq:engeequu}
  D_{\mbox{\tiny LS}}(p\|q)=\sum{}p_{i}\ln\frac{p_{i}}{\sum{}p_{k}}-\sum{}p_{i}\ln\frac{q_{i}}{\sum{}q_{k}}
\end{equation}
In the case of BS, it is
\begin{equation}
  \label{eq:aphooghi}
  D_{\mbox{\tiny BS}}(p\|q)=\sum{}p_{i}\left[\sum_{j}\left(\frac{q_{j}}{\sum{}q_{k}}-\delta_{ij}\right)^{2}-\sum_{j}\left(\frac{p_{j}}{\sum{}p_{k}}-\delta_{ij}\right)^{2}\right]
\end{equation}
where $\delta_{ij}$ is the Kronecker delta. In either case, for LS or
BS, the divergence is the difference between the expected loss of
reported $q$ when $p$ is the true distribution and the entropy of
$p$, which is the expected loss of reported $p$ when $p$ is the true
distribution.

Define
\begin{equation}
  \label{eq:kiegiqui}
  F^{\ast}(\hat{x})=\sup_{x\in\mathcal{X}}\left\{\langle{}\hat{x},x\rangle{}-F(x)\right\}
\end{equation}
Let $G(x)=\langle{}\hat{x},x\rangle{}-F(x)$. Then the supremum is
reached where the gradient of $G(x)$ vanishes.
\begin{equation}
  \label{eq:ahkoobou}
  \frac{\partial{}G}{\partial{}x_{i}}=\frac{\partial}{\partial{}x_{i}}\left(\sum{}x_{i}\frac{\partial{}F}{\partial{}x_{i}}-F(x)\right)
\end{equation}


\section{Leonard Savage: Elicitation of Personal Probabilities and Expectations}
\label{section:euzaevoe}

The quarrel between Brier score and Log score roughly maps onto the
quarrel between difference and ratio in the debate about evidence.
\begin{itemize}
\item If loss is a function of difference discrepancy $x-r$, then this
  condition is equivalent to symmetry and only linear variations of
  the Brier score fulfill it.
\item If loss is a function of ratio discrepancy $r/x$, then this condition
  is equivalent to horizon and only linear variations of the Log
  score fulfill it.
\item If, more generally, loss is a function of $g(r)-g(x)$, then only
  the Log score or the Brier score fulfill it (789).
\end{itemize}
Savage helpfully uses the terminology of the trichotomy (win, loss,
tie). I put screen shots of Trump/Clinton predictions by nytimes/538
in the folder \texttt{barney/learning/dissertation/chapters/GeometryOfReason}.

\section{Tilmann Gneiting and Matthias Katzfuss: Probabilistic Forecasting}
\label{section:shaemabi}

Nice diagram of a probabilistic forecast on page 127 (Bank of
England). The goodness of a probabilistic forecast is measured by the
extent to which realizations are distinguishable from from random
draws from predictive distributions (127). Propriety is equivalent to
Bregman form is attributed to Savage 1971 on page 136.

\section{Claude Shannon: Mathematical Theory of Communication}
\label{section:aehooboo}

Go straight to section 6 on page 10. The proof is in appendix 2 on
page 28.
\begin{enumerate}
\item $H$ should be continuous in the $p_{i}$
\item If all the $p_{i}$ are equal, $p_{i}=1/n$, then $H$ should
  be a monotonic increasing function of $n$. With equally likely
  events there is more choice, or uncertainty, when there are more
  possible events.
\item If a choice be broken down into two successive choices, the
  original $H$ should be the weighted sum of the individual values
  of $H$.
\end{enumerate}
Here is an example for the last requirement,
\begin{equation}
  \label{eq:ohfaexoo}
  H\left(\frac{1}{2},\frac{1}{3},\frac{1}{6}\right)=H\left(\frac{1}{2},\frac{1}{2}\right)+\frac{1}{2}H\left(\frac{2}{3},\frac{1}{3}\right)
\end{equation}
The only $H$ satisfying the three above assumptions is of the form
\begin{equation}
  \label{eq:eogohupi}
  H=-K\sum_{i=1}^{n}p_{i}\log{}p_{i}
\end{equation}

\section{John McCarthy: Measures of the Value of Information}
\label{section:bohfiebe}

McCarthy uses the expression ``keep the forecaster honest'' (654).

\textbf{Theorem 1.} A payoff rule keeps the forecaster honest if and only if
\begin{equation}
  \label{eq:hievaime}
  f_{i}(q)=\frac{\partial}{\partial{}q_{i}}f(q)
\end{equation}
is a convex function of $q$ which is homogeneous of the first degree.
$f_{i}{q}$ is the reward for the occurrence of the $i$-th event. This
is terrible notation. $f_{i}(p)$ is $S(x_{i},P)$ and $f(p)$ is $H(P)$.
McCarthy omits the proof. The expectation of an
honest forecaster is then
\begin{equation}
  \label{eq:kohrupov}
  \sum{}p_{i}f_{i}(p)=f(p)
\end{equation}

I.J. Good considered the problem of paying the forecaster with the
restriction that $f_{i}(q)=F(q_{i}$, i.e.\ the payoff depends only on
the probability assigned to the event which actually occurred. He
showed that putting
\begin{equation}
  \label{eq:oquohshu}
  F(x)=A\ln{}x+B
\end{equation}
keeps the forecaster honest, and Gleason (unpublished) showed that
this is the only $F(x)$ which does. (This is McCarthy verbatim, 654.)

I played this through for the Brier Score and the Log Score in
Schmierbuch 2388f. Indeed,
\begin{equation}
  \label{eq:leisooth}
  \sum_{j}q_{j}\frac{\partial{}S(x_{j},Q)}{\partial{}q_{i}}=0
\end{equation}
for the Brier Score $S(x_{i},Q)=1-2q_{i}+\sum_{j}q_{j}^{2}$. However,
for the Log Score $S(x_{i},Q)=\ln{}q_{i}$ I get
\begin{equation}
  \label{eq:iebaingu}
  \sum_{j}q_{j}\frac{\partial{}S(x_{j},Q)}{\partial{}q_{i}}=1
\end{equation}

McCarthy's theorem says (\ref{eq:iebaingu}) implies that the Log Score
is not proper. What gives? Answer: Hendrickson and Buehler identify
this problem on page 1918. The derivative needs to be taken with
respect to $D$ (the convex cone of $\mathcal{P}$), not $\mathcal{P}$.

Note that
\begin{equation}
  \label{eq:eetheaga}
  \sum_{j}q_{j}\frac{\partial{}S(x_{j},Q)}{\partial{}q_{i}}=0
\end{equation}
looks suspiciously like the LHS in Euler's Homogeneous Function
Theorem in
\begin{alltt}
  http://mathworld.wolfram.com/EulersHomogeneousFunctionTheorem.html
\end{alltt}

The intuitive content of the convexity restriction is that it is
always a good idea to look at the outcome of an experiment if it is
free (655). 

\section{Arlo D. Hendrickson and Robert J. Buehler: Proper Scores for
  Probability Forecasters}
\label{section:eibualiu}

\subsection{Inner Products}
\label{subsection:rioxesho}

HB have an interesting suggestion. $p=(p_{1},{\ldots},p_{n})$ (the
credence function) is as usual an $n$-dimensional vector, but now so
is $f(q)=(S(x_{1},q),{\ldots},S(x_{n},q))$, the score of $q$. This
move turns the expectation of $q$ (the prediction) with respect to
true world $p$ into an inner product $p\cdot{}f(q)$. Propriety demands
\begin{equation}
  \label{eq:aigureex}
  H(p)=p\cdot{}f(p)\geq{}p\cdot{}f(q)
\end{equation}
for all $p,q$ in order to discourage hedging. See the comedy of errors
in section~\ref{subsection:iegaecee}.

\subsection{Definition 2.1 of Subgradients}
\label{subsection:zaegheef}

A subgradient, in my understanding, is a linear real-valued function $L$ which, if a
real-valued function $f$ is convex on a neighbourhood $U$ of
$x\in\mathbb{R}^{n}$, is less than $f$, i.e.\ $L(y)\leq{}f(y)$ on $U$.
HB define as follows:
\begin{equation}
  \label{eq:wiokaeku}
  f(y)\geq(y-x)L(y)+f(x)
\end{equation}
If all of this is one-dimensional, it's straightforward enough.
\begin{equation}
  \label{eq:hooxajee}
  \frac{f(y)-f(x)}{y-x}
\end{equation}
is the slope of the secant line, and for a differentiable, convex
function $f:\mathbb{R}\rightarrow\mathbb{R}$ the only way
(\ref{eq:wiokaeku}) can be true is if $L(y)=f'(y)$. When $n\geq{}2$,
$L(y)$ is presumably the vector of partial derivatives and the product
is the inner product. So, if $f$ is convex and differentiable,
\begin{equation}
  \label{eq:iiquocah}
  f(y)-f(x)\geq\sum(y_{i}-x_{i})\frac{\partial{}f}{\partial{}y_{i}}(y)
\end{equation}
I've always been puzzled that the slopes along the coordinate axes are
simply added, but this appears to be true also for the product rule in
higher dimensions, see (\ref{eq:jeeyongu}). If $f$ is differentiable,
then the subgradient uniquely equals the gradient (see Rockafellar,
theorem 25.1).

\subsection{Euler's Theorem}
\label{subsection:eephuupe}

Theorem 2.1 tells us that if a function $f$ has a subgradient
everywhere on a convex set $D$, then $f$ is convex on $D$. When HB
refer to Euler's Theorem, they mean Euler's Homogeneous Function
Theorem, see
\begin{alltt}
http://mathworld.wolfram.com/EulersHomogeneousFunctionTheorem.html
\end{alltt}

Theorem 2.2 is basically Euler's Theorem, but HB give an elegant
proof. For the proof, note that
\begin{equation}
  \label{eq:quohnahj}
  \frac{\lambda^{r}-1}{\lambda-1}=\lambda^{r-1}+{\ldots}+\lambda^{1}+\lambda^{0}
\end{equation}
As $\lambda\rightarrow{}1$, (\ref{eq:quohnahj}) goes to $r$ because
there are $r$ summands.

\subsection{Brier Score and Log Score}
\label{subsection:eesheing}

HB give a nice explanation why a superficial differentiation of
the entropy function for the log score doesn't give you the log
score, see page 1918.

I checked the following using pen and paper first, sage second.
The Brier score $f:\mathbb{R}^{n}\rightarrow\mathbb{R}^{n}$ is for
$I=\{1,{\ldots},n\}$
\begin{equation}
  \label{eq:xeezaise}
  f(x)=\left(2x_{i}-1-\sum_{k=1}^{n}x_{k}^{2}\right)_{i\in{}I}
\end{equation}
where $x\in\mathcal{P}$. It is interesting that in HB the scoring
rule is restricted to the set of probability vectors
$\mathcal{P}$. The entropy function
$H:\mathbb{R}^{n}\rightarrow\mathbb{R}$ is defined on the convex
cone $\mathcal{D}$. It is
\begin{equation}
  \label{eq:aejujuez}
  H(y)=H(\lambda{}x)=\lambda{}H(x)=\lambda{}x\cdot{}f(x)
\end{equation}
for $y\in\mathcal{D},x\in\mathcal{P}$, and
$\lambda=\sum{}y_{i}$. So defined, it is indeed true
that
\begin{equation}
  \label{eq:ieghaico}
  \frac{\partial{}H(y)}{\partial{}y_{i}}=f_{i}(y)=2x_{i}-1-\sum_{k=1}^{n}x_{k}^{2}
\end{equation}
when $y\in\mathcal{P}$, but the derivative of $H$ must be taken with
respect to $\mathcal{D}$. For the log score, the corresponding
functions are
\begin{equation}
  \label{eq:noofitho}
  f(x)=\left(\log{}x_{i}\right)_{i\in{}I}
\end{equation}
and, indeed
\begin{equation}
  \label{eq:oofoothi}
  \frac{\partial{}H(y)}{\partial{}y_{i}}=f_{i}(y)=\log{}y_{i}
\end{equation}
There is some sage code commented in the source code of \texttt{supplement.tex}.
% sage: x=var('x')
% sage: y=var('y')
% sage: z=var('z')
% sage: Hls(x,y,z)=x*log((x/(x+y+z)))+y*log((y/(x+y+z)))+z*log((z/(x+y+z)))
% sage: gls(x,y,z)=diff(Hls(x,y,z),x)

\section{Gneiting and Raftery: Strictly Proper Scoring Rules,
  Rediction, and Estimation}
\label{section:faokuoju}

Generally nice summary. Bregman divergences come into play once the
sample space is finite and the entropy function is sufficiently smooth
(361). More details, I believe, are in Predd et al (2009). 

Note the use of hedging strategies on page 362. 

\section{Evgeni Ovcharov: Proper Scoring Rules and Bregman Divergences}
\label{section:eitooshe}

\subsection{A Comedy of Errors}
\label{subsection:iegaecee}

It seems to me that there is a mistake in Ovcharov. Definition 2.2
says that a scoring rule $S$ is called proper if it maximizes its
expected score at the true density,
\begin{equation}
  \label{eq:iemieder}
  p\cdot{}S(p)=\max_{q\in\mathcal{P}}p\cdot{}S(q)
\end{equation}
However, it seems to me that this should say
\begin{equation}
  \label{eq:ieweiphi}
  p\cdot{}S(p)=\max_{q\in\mathcal{P}}q\cdot{}S(p)
\end{equation}
The truth needs to vary, not the prediction!

This is crazy. Ovcharov (5) and Hendrickson/Buehler (1916) seem to agree; but
Dawid et al and Gneiting/Raftery have it my way. Let a chance-world be
such that if it is known the chances are known but not the outcomes.

\begin{description}
\item[Ovcharov/Hendrickson/Buehler] There is one true chance-world. I
  want to avoid that a forecaster will choose a forecast that is
  different from what she knows to be true. This is not realistic.
  Forecasters don't know which chance-world is true.
\item[Dawid/Gneiting/Raftery] There are many possible chance-worlds,
  an unknown one of which is true. I want to avoid that a forecaster
  will choose a forecast that does not reflect her best estimate. For
  example, a TV weather forecaster may deflate POP in order to boost
  TV ratings. In this case, the reward $p\cdot{}f(q)$ ($p$ is the
  forecaster's best estimate, $q$ is the deflated estimate) is greater
  than the reward $p\cdot{}f(p)$; which is what we don't want. But
  that's what Ovcharov/Hendrickson/Buehler are saying.
\end{description}

This is all a comedy of errors. The only person who's had this wrong
is I. Both Ovcharov/Hendrickson/Buehler and Dawid/Gneiting/Raftery
clearly state that the true world stays fixed and the prediction
varies. 

\section{Leszek Wronski: Belief Update Methods and Rules}
\label{section:wiebahyu}

\section{Dawid, Lauritzen, Parry: Proper Local Scoring Rules on
  Discrete Sample Spaces}
\label{section:ahtiesei}

Let $P=(p_{1},{\ldots},p_{n})$ be a probability distribution over $n$
mutually exclusive and collectively exhaustive events $x_{i}$. Let $Q$
be the true (probability) distribution. Then $S(x_{i},P)$ is the loss
suffered by the agent espousing $P$ when $x_{i}$ is true. Define
\begin{equation}
  \label{eq:nahcaece}
  S(Q,P)=\sum_{i=1}^{n}q_{i}S(x_{i},P)
\end{equation}
which is the expected loss of $P$ with respect to $Q$. Dawid et al
have nature first, forecaster second. Proper scoring rules require for
$H(P)=\inf_{Q\in\mathcal{P}}S(Q,P)$
\begin{equation}
  \label{eq:moozeosh}
  H(P)=S(P,P)
\end{equation}
$H$ is the entropy function. (\ref{eq:moozeosh}) encourages honesty.
It prohibits the following scenario (which has been called `hedging').
A forecaster's best guess at the true probability distribution $Q$ is
$P'$ (with $P'\neq{}P$), but she knows that $S(P',P)<S(P',P')$, so
instead of asserting $P'$ she asserts $P$.

The divergence function is
\begin{equation}
  \label{eq:cienaeyo}
  d(P,Q)=S(P,Q)-S(Q,Q)
\end{equation}
Notice that strict propriety requires that $d(P,Q)>0$ iff $P\neq{}Q$
(i.e.\ that $d$ is positive-definite). 

The Brier score is
\begin{equation}
  \label{eq:xoophoot}
  S_{B}(x_{i},P)=1-2p_{i}+\sum_{j=1}^{n}p_{j}^{2}
\end{equation}
The summand 1 is sometimes omitted (as in Gneiting and Raftery,
example 2.1, but not in Pettigrew). Consequently,
\begin{equation}
  \label{eq:phiengae}
  S_{B}(P,Q)=\sum_{i=1}^{n}q_{i}\left(1-2p_{i}+\sum_{j=1}^{n}p_{j}^{2}\right)=1-2(P\cdot{}Q)+\sum_{i=1}^{n}p_{i}^{2}
\end{equation}
where $P\cdot{}Q$ is the dot product. Note that
\begin{equation}
  \label{eq:thequaev}
  \frac{\partial}{\partial{}p_{k}}=2p_{k}-2q_{k}
\end{equation}
so that, indeed, $\inf_{Q\in\mathcal{P}}S(Q,P)=S(P,P)$.

This means that
\begin{equation}
  \label{eq:aechahso}
  H_{B}(P)=S_{B}(P,P)=1-\sum_{i=1}^{n}p_{i}^{2}
\end{equation}
Again, the summand 1 is omitted accordingly. The divergence function
is the squared Euclidean distance
\begin{equation}
  \label{eq:goweepip}
  d_{B}(P,Q)=\sum_{i=1}^{n}(p_{i}-q_{i})^{2}
\end{equation}

Now let's run through this for the logarithmic score (see Schmierbuch 2415).
\begin{equation}
  \label{eq:ohngooth}
  S(x_{i},P)=-\ln{}p_{i}
\end{equation}
The entropy function is the Shannon entropy
\begin{equation}
  \label{eq:ahsainie}
  H(P)=-\sum_{i=1}^{n}p_{i}\ln{}p_{i}
\end{equation}
The divergence function is 
\begin{equation}
  \label{eq:ohbaibah}
  d(P,Q)=S(P,Q)-S(Q,Q)=\sum_{i=1}^{n}(q_{i}-p_{i})\ln{}q_{i}
\end{equation}
which, disappointingly, is not the Kullback-Leibler divergence
\begin{equation}
  \label{eq:oopeesae}
  D_{\mbox{\tiny KL}}(Q,P)=S(Q,P)-S(Q,Q)
\end{equation}
Always remember that by convention the second argument of the
Kullback-Leibler divergence is the prior probability. 

The logarithmic score is local in the sense that $S(x_{i},P)$ depends
only on $p_{i}$. The Brier score is not local in this sense (see Dawid
et al, 597f; I.J. Good addresses this as well). 

\section{Richard Pettigrew: Accuracy and the Laws of Credence}
\label{section:pettigrew}

\subsection{Measuring Accuracy: A New Account}
\label{subsection:xiphievi}

\subsubsection{Using Lagrange Multipliers to Find a Dominant
  Probability Distribution}
\label{subsubsection:hiewohph}

Let $\mathcal{F}$ be an algebra of propositions. There may be other
algebras of propositions, but it's easy to construct one using
events that are logically unrelated in the sense that they do not
entail one another, although they may be probabilistically dependent
on each other. Let these events be $E_{1},{\ldots},E_{n}$. Then the
algebra contains $2^{2^{n}}$ elements, $2^{n}$ atomic events of the
form
\begin{equation}
  \label{eq:mahpoogh}
  \omega_{i}=(\urcorner)E_{1}\cap{\ldots}\cap(\urcorner)E_{n}
\end{equation}
for $i=1,{\ldots},2^{n}$ and then combinations of the $\omega_{i}$
such as
\begin{equation}
  \label{eq:aikaerei}
  r_{j}=\omega_{3}\cup\omega_{6}\cup\omega_{12}
\end{equation}
for $j=1,{\ldots},2^{2^{n}}$. If there is only one event $A$, the
algebra is 
\begin{equation}
  \label{eq:shiegaev}
  \mathcal{F}=\{\emptyset,A,\urcorner{}A,\Omega\}
\end{equation}
where $\emptyset$ is the conjunction of all other propositions in the
algebra and $\Omega$ is the disjunction of all other propositions in
the algebra.

My concern is that Joyce and Pettigrew only evaluate inaccuracy on
atomic events. Here is my proof that for atomic events, there is
indeed always a probability function which will dominate a credence
function unless that credence function is itself a probability
function. Let $m=2^{n}$ be the number of atomic events and $q$ a
credence function defined on them with $\sum{}q_{i}=Q$. Then
(beginning on page 2398 in Schmierbuch) I want to find a $p$ with
$\sum{}p_{i}=1$ that dominates $q$, meaning that $p$'s inaccuracy
measured by the Brier score is less than $q$'s inaccuracy, no matter
which one of the $\omega_{i}$ is true. To find such a $p$, I look for
the one that is closest to $q$. Define the Lagrangian
\begin{equation}
  \label{eq:ohyaihoh}
  L(p)=\sum_{i=1}^{m}(p_{i}-q_{i})^{2}+\lambda\left(\left(\sum_{i=1}^{m}p_{i}\right)-1\right)
\end{equation}
Differentiate with respect to each $p_{i}$ and resolve the constraint
for the following result
\begin{equation}
  \label{eq:aezedoop}
  p_{k}=q_{k}-\frac{1}{m}(Q-1)
\end{equation}
It turns out that the difference $I(q,w)-I(p,w)$ is a function
$J_{(m,w)}(Q)$ of the sum of the $q_{i}$'s, but not the individual
$q_{i}$'s themselves. Let $w_{i}=1$ iff $\omega_{i}$ is true
and $w_{i}=0$ iff $\omega_{i}$ is false. Recall that
\begin{equation}
  \label{eq:baviedam}
  I(q,w)=\sum_{i=1}^{m}\left(q_{i}-w_{i}\right)^{2}
\end{equation}
(although this is a simplification of the algebra that I will cast
doubt on later). Then
\begin{equation}
  \label{eq:einaicoo}
J_{(m,w)}(Q)=\frac{1}{m}(Q-1)^{2}
\end{equation}
This result depends on
\begin{equation}
  \label{eq:tiduahom}
\sum_{i=1}^{m}w_{i}=1  
\end{equation}
which is true because the atomic events are pairwise disjoint and
their disjunction is $\Omega$. (\ref{eq:einaicoo}) tells us that a
dominating $p$ can be found for each $q$ that is not itself a
probability function, i.e.\ $Q=1$. It will be interesting to compare
this proof to Pettigrew's proof. On page 80, he refers to theorems
I.A.2, I.D.5, and I.D.7, which together establish theorem 4.3.4, and
theorem 4.3.4 makes the above dominance claim.

\subsubsection{Tetrahedrons and $n$-Simplices}
\label{subsubsection:ofiedeeb}

Indeed, a couple of days later, it was very interesting! De Finetti
uses convex hulls to show that probability functions and \emph{only}
probability functions are convex combinations of worlds. Consider
two events $A,B$. 
\begin{equation}
  \label{eq:ongeefoo}
  \begin{array}{rcl}
  \omega_{0}&=&\urcorner{}A\cap\urcorner{}B \\
  \omega_{1}&=&\urcorner{}A\cap{}B \\
  \omega_{2}&=&A\cap\urcorner{}B \\
  \omega_{3}&=&A\cap{}B
  \end{array}
\end{equation}

Let the $r_{j}$ be as in (\ref{eq:aikaerei}), e.g.\
$r_{0}=\emptyset,r_{14}=\urcorner(\urcorner{}A\cap\urcorner{}B),r_{15}=\Omega$.
(I have been using $\urcorner$ for the set complement, which is a bit
of notational mishmash between events and propositions.) Then the
$r_{j}$ are an exhaustive list of the propositions that you can form
using $A,B$. There are four possible worlds, one for each $\omega_{i}$
to be true. I will write the vectors in matrix form to save space.
Here are the four possible worlds, indicating whether $r_{j}$ is false
or true. 

\begin{equation}
  \label{eq:aicooroh}
  W_{0}=\left[
    \begin{array}{cccc}
      0 & 0 & 0 & 0 \\
      0 & 0 & 0 & 0 \\
      1 & 1 & 1 & 1 \\
      1 & 1 & 1 & 1 \\
    \end{array}\right]
\end{equation}

\begin{equation}
  \label{eq:johruave}
  W_{1}=\left[
    \begin{array}{cccc}
      0 & 0 & 0 & 0 \\
      1 & 1 & 1 & 1 \\
      0 & 0 & 0 & 0 \\
      1 & 1 & 1 & 1 \\
    \end{array}\right]
\end{equation}

\begin{equation}
  \label{eq:eivohcoy}
  W_{2}=\left[
    \begin{array}{cccc}
      0 & 0 & 1 & 1 \\
      0 & 0 & 1 & 1 \\
      0 & 0 & 1 & 1 \\
      0 & 0 & 1 & 1 \\
    \end{array}\right]
\end{equation}

\begin{equation}
  \label{eq:kohzahsi}
  W_{3}=\left[
    \begin{array}{cccc}
      0 & 1 & 0 & 1 \\
      0 & 1 & 0 & 1 \\
      0 & 1 & 0 & 1 \\
      0 & 1 & 0 & 1 \\
    \end{array}\right]
\end{equation}

Z{\"a}hneknirschend r{\"a}ume ich es ein, that de Finetti succeeds in
showing that all and only probability functions are convex
combinations of these four worlds,

\begin{equation}
  \label{eq:yiehieli}
  p=\lambda{}W_{3}+\mu{}W_{2}+\nu{}W_{1}+(1-\lambda-\mu-\nu)W_{0}
\end{equation}

The probability distributions are a three-dimensional shape in a
sixteen-dimensional space. More generally, for $n$ events, the
probability distributions form a $(2^{n}-1)$-dimensional simplex in a
$(2^{2^{n}})$-dimensional space. For two events, it is a tetrahedron in
$\mathbb{R}^{16}$; for three events, it is a 7-simplex in
$\mathbb{R}^{256}$. So, I need to get off the triangle/simplex
bandwagon that I was on when I wrote the Geometry of Reason and get on
the tetrahedron/$n$-simplex bandwagon!

BdF claims that for a convex set and an arbitrary point $x$, we can always
find a point $y$ that is an element of the convex set which is at least as
close to all the other elements of the convex set as $x$ is. Is this
true when we use the logarithmic divergence measure?

\subsubsection{Differential Equations for Bregman Divergences}
\label{subsubsection:suthahco}

Let $D$ be a divergence with
$D:[0,1]^{n}\times[0,1]^{n}\rightarrow[0,\infty]$ and $D(x,y)\geq{}0$,
equality iff $x=y$. Then $D$ is an additive Bregman divergence iff the
following conditions are met:
\begin{equation}
  \label{eq:oengaegh}
  D(x,y)=\sum_{i=1}^{n}d(x_{i},y_{i})
\end{equation}
where $d$ is a one-dimensional divergence and generated by a smooth,
bounded, strictly convex function $f$ such that for all $a,b\in[0,1]$
\begin{equation}
  \label{eq:thohauma}
  d(a,b)=f(a)-f(b)-f'(b)(a-b)
\end{equation}
$d(a,b)$ is effectively the difference at $a$ between $f$ and the
first Taylor expansion of $f$ with respect to $b$. Finding $f$ for a
given divergence $d$ is a differential equation problem which I have
posed on Math Stackexchange at \texttt{https://math.stackexchange.com/questions/2586331/ordinary-}
\newline
\texttt{differential-equation-with-deviating-distributed-retarded-arguments}

Note that there is good material in Predd et.\ al.\ (2009) about this.

\subsection{Bronfman Objection}
\label{subsection:bronfman}

This is where Pettigrew makes an argument for symmetry. It would be a
good entry point for my criticism. It is an epistemic argument for
symmetry in the sense that its validity depends on our current
possibly deficient epistemic state.

Reading notes:

\begin{description}
\item[page 2] RP says No Drop is ``if agent has opinion set $\{A,B\}$
  and $A$ entails $B$ then rationality requires that
  $c(A)\leq{}c(B)$.'' However, rationality does not require that the
  agent knows that $A$ entails $B$. Is the agent logically omniscient?
\item[page 4] RP defines the squared Euclidean distance in terms of
  atomic propositions, but keeps saying all propositions. Which one is
  it?
\item[page 26] The Brier score is proper; is the Kullback-Leibler
Divergence proper?
\item[page 17] Theorem 1.0.2 can't be right the way it stands because
  it privileges normalized probability functions.
\item[ibidem] High-level criticism, not just of Pettigrew, but also of
  Joyce and de Finetti. When they prove de Finetti's theorem, do they
  take into account that a credence function is completely free in
  choosing any credence for any proposition in the algebra, not just
  the atomic propositions? If there is one event $E$, then the algebra
  consists of 4 elements, $\{\emptyset, A, not-A, \Omega\}$. If there
  are two events $A$ and $B$, then there are four atomic propositions
  which determine any probability distribution: A and B, A and not-B,
  not-A and B, not-A and not-B. The algebra consists of any
  combination of these atomic propositions connected by OR, for how to
  put this in code see definetti.jl. If $n$ is the number of events
  under consideration, $2^n$ is the number of atomic propositions and
  $2^(2^n)$ is the number of propositions in the algebra.
  Probabilities are only free over the atomic propositions; credences
  are free over the whole algebra. There are several things I want to
  know: are there credences that are not Brier-score-dominated by a
  single probability distribution given this more complicated state of
  affairs? Does de Finetti's proof address this more complicated state
  of affairs? Are there credences (which must be positive to be
  measurable by DKL) that are not DKL-dominated by a single
  probability distribution at each possible world?
\item[section 4.1] Pettigrew claims that there are no irreducible
  global features of inaccuracy. I agree. This, however, does not mean
  that global inaccuracy must be the sum of local inaccuracies. It
  only means that global inaccuracy is a function of local
  inaccuracies, and there are many other functions beside addition
  that would give us the local veritism that Wormwood wants.
\item[page 51] Here is a problem with DKL and local inaccuracy. Let my
  credence that a coin toss is H be 0.52. The world w1 at which we
  measure my credence's local inaccuracy with respect to this
  proposition is such that H is true. Then a local version of
  inaccuracy for DKL is (?) $1\cdot\log(1/0.52)$ or $-\log(0.52)$. So
  far so good, except that it subscribes to regularity in the sense
  that my inaccuracy is infinite if I have an extreme credence of
  zero. However, if at w2 not-H is true, then my local inaccuracy is
  $0\cdot\log(0/0.52)$, which is conventionally taken to be zero. I
  wouldn't have been inaccurate at all! Furthermore, my local
  inaccuracy for the logically equivalent proposition not-H, in which,
  say, I have credence 0.48, is completely different! Do these
  problems go away if I model credences as manifolds and apply the
  Fisher metric rather than modeling them using Cartesian coordinates
  and applying DKL? What Wormwood calls a one-dimensional divergence
  does not appear to be straightforward for logarithmic measures.
\item[page 83] Why is $\mathcal{B}_{\mathcal{F}}$ restricted to credence
  functions that take values between 0 and 1? This may help me because
  DKL can't apply to negative numbers or zero. Pettigrew is consistent
  here with his definition of credence functions on page 16,
  Definition 1.0.1.
\item[page 83] What is $\mathcal{W}_{\mathcal{F}}$? I am assuming it is
  a set of worlds which corresponds to whether the propositions in
  $\mathcal{F}$ are true or not. $\mathcal{F}$, according to Pettigrew, is
  not necessarily an algebra. A credence function $c$, however, is
  only a probability function if it can be extended to a credence
  function that is a probability function on an algebra. There is some
  circularity in P's definition of what a probability function is in
  Definition 1.0.1.
\item[ibidem] The meat is all in theorem I.A.2. Let's see if we can
  poke any holes. Here is the only place I can think of: if $p$ and
  $p'$ are probability functions, then so is
  $\lambda{}p+(1-\lambda)p'$.
\end{description}

\section{Bernhard Schutz: Geometrical Methods of Mathematical Physics}
\label{section:schutz}

\subsection{Some Basic Mathematics}
\label{subsection:veecheic}

On page 15, show that $d''(x,y)$ is not a norm.

\subsection{Differentiable Manifolds and Tensors}
\label{subsection:vieghaem}

On page 28, cover the interior of the annulus by a single coordinate
patch.

\subsubsection{Vectors and Vector Fields}
\label{subsubsection:eiciexie}

Explain (2.2) on page 32. Schutz' $\lambda$ is confusing to me. It
appears to be a function from a closed interval
$[a,b]\subset\mathbb{R}$ into the manifold $M$ (a curve). $f$ is
$C^{\infty}(M)$, so $f:U\rightarrow\mathbb{R}$. $g$ is just a
coordinate map, what Jeff Lee calls $x$, and goes from $U$ to
$\mathbb{R}^{n}$. $g:U\rightarrow\mathbb{R}$. This makes no sense to
me. What the hell is 
\begin{equation}
  \label{eq:phohvuoj}
  \frac{dg}{d\lambda}
\end{equation}
if $g$ is defined on $U$ and goes into $\mathbb{R}^{n}$?

Answer: you are mistaking figure 2.9 to apply to section 2.7 instead
of section 2.6. $g$ is not a coordinate map in section 2.7. It looks
like the $x^{i}$ determine the curve. In Jeff Lee's notation, let
$\gamma:(a,b)\rightarrow{}V$ ($(V,y)$ is the usual open neighbourhood
in $M$ plus coordinate map pair). Then Schutz'

\begin{equation}
  \label{eq:eenooroh}
(x^{1}(\lambda),\ldots,x^{n}(\lambda)) 
\end{equation}

corresponds to Lee's
$(y\circ{}\gamma)(\lambda)$ in $\mathbb{R}^{n}$. Schutz'
$f:\mathbb{R}^{n}\rightarrow\mathbb{R}$ corresponds to 

\begin{equation}
  \label{eq:neilophi}
  \hat{f}:\xi\in\mathbb{R}^{n}\rightarrow{}f(y^{-1}(\xi))
\end{equation}

in Lee. I derived (2.2) in Schutz with a complicated translation table
between Lee and Schutz on page 2256 in Schmierbuch. Write $g$ in
Schutz instead as $\hat{f}\circ{}x$, where $\hat{f}$ is as defined in
(\ref{eq:neilophi}) and $x$ is as defined in (\ref{eq:eenooroh}),
unfortunately mixing Schutz and Lee, but (2.2) pretty handily follows.

Here is an example for the chain rule and multivariable
differentiation. Let

\begin{equation}
  \label{eq:eengoogu}
  \begin{array}{rcl}
    f(x)&=&\left(x^{2},\frac{x}{2}\right) \\
    g(w,z)&=&3w-z^{2}
  \end{array}
\end{equation}

Then $(g\circ{}f)(x)=\frac{11}{4}x^{2}$ and therefore
$(g\circ{}f)'(x)=\frac{11}{2}x$. According to the chain rule

\begin{equation}
  \label{eq:ahhoiqua}
  \frac{d(g\circ{}f)}{dx}=\sum\frac{df^{i}}{dx}\frac{\partial(g\circ{}f)}{\partial{}f^{i}}=2x\cdot{}3+\frac{1}{2}\cdot\left(-2\cdot\frac{x}{2}\right)=\frac{11}{2}x
\end{equation}

This is what is behind Schutz' equation (2.2) on page 32.

\subsubsection{Exercise 2.1}
\label{subsubsection:veecheic}

On page 44.

\section{Michael K. Murray and John W. Rice:\newline Differential Geometry and
Statistics}
\label{murrayrice}

Note that there is an informative article on exponential families and
categorical distributions in Wikipedia. It suggests as parameters for
the categorical distribution

\begin{equation}
  \label{eq:eemaepie}
  \left(\ln\frac{p_{1}}{p_{k}},{\ldots},\ln\frac{p_{k-1}}{p_{k}},0\right)^{\intercal}.
\end{equation}

I wonder if the last parameter is necessary, since it's always zero.

Let $P$ be the parametric family of normal distributions. Let
$q=\mathcal{N}(0,1)$ (the origin) and $p=\mathcal{N}(7,9)$. The
density for the normal distribution is

\begin{equation}
  \label{eq:balahghe}
  p(\mu,\sigma)=\frac{1}{\sqrt{2\pi}\sigma}\exp\left(\frac{-(x-\mu)^{2}}{2\sigma^{2}}\right)
\end{equation}

Expressed as an exponential family, this is

\begin{equation}
  \label{eq:gexahmai}
  p(\theta^{1},\theta^{2})=\exp\left(\theta^{2}x^{2}+\theta^{1}x^{1}-K(\theta)\right)
\end{equation}

$x^{2}$ is a second random variable (using differential geometry's
habit of indexing vector components on the top right), but in the case
of the normal distribution it is the first random variable squared,
which makes for a hell of notational confusion.

For $q$ and $p$ as above, the parameters are $(-0.5,0)$ and
$(-1/18,7/9)$, respectively. The origin $q$ is not at $(0,0)$, which
also creates confusion. The vector $\vec{w}$ associated with $p$, for
example, so that

\begin{equation}
  \label{eq:aelayele}
  q\oplus{}\vec{w}=p
\end{equation}

where $q\oplus{}\vec{w}=exp(\vec{w}x)q(x)$ ($x$ is $(x^{1},x^{2})$),
is

\begin{equation}
  \label{eq:iiwiriew}
  \vec{w}^{\intercal}=\left(\frac{4}{9},\frac{7}{9}\right).
\end{equation}

However, $q\oplus{}\vec{w}=p^{\ast}$ such that $[p^{\ast}]=[p]$, where
$[p]$ is the equivalence class of measures up to scale. In other
words, $p^{\ast}$ is $p$ once it is normalized (2153).

For the normal distribution,

\begin{equation}
  \label{eq:ooleengu}
  \theta^{1}=-\frac{1}{2\sigma^{2}},\hspace{.5in}\theta^{2}=\frac{\mu}{\sigma^{2}},\hspace{.5in}K(\theta)=\frac{1}{2}\ln\left(-\frac{\pi}{\theta^{1}}\right)-\frac{(\theta^{2})^{2}}{4\theta^{1}}
\end{equation}

On (2155f) I wonder if probability distributions/densities generally
wouldn't be better expressed by equivalence classes of measures rather
than normalized measures. The following appear to be privileged once
you use the exponential parameters rather than simply the
probabilities of the categories:

\begin{itemize}
\item sufficient statistics
\item conjugate priors
\item maximum entropy derivation (see Wikipedia)
\end{itemize}

The log-likelihood of $\mathcal{N}(0,1)$ is 0 because

\begin{equation}
  \label{eq:ohghaeng}
  q=q\oplus{}0=e^{0}q
\end{equation}

The log-likelihood of $\mathcal{N}(7,9)$ is 

\begin{equation}
  \label{eq:aiwahief}
  \ln\frac{1}{3}+\frac{1}{18}(8x^{2}+14x-49)
\end{equation}

Murray and Rice appear to get this wrong, assuming that the
coordinates of the origin $q$ are $(0,0)$ so that the log-likelihood
of $p$ would be $\theta_{p}x-K(\theta_{p})$, when really it is
$\alpha_{p}x-K(\theta_{p})$ with $\theta_{q}+\alpha_{p}=\theta_{p}$
(2160).

Murray and Rice make a momentous claim on page 16:

\begin{quote}
  Hence $P$ is an exponential family if and only if $\tilde{P}$ is an
  affine subspace of $\mathcal{M}$.
\end{quote}

I could not track with this claim and started reading Jeff Lee's book
in order to understand this better (beginning of August, 2017; notes
for Murray and Rice end on (2171) and begin for Lee on (2172)). I
should note (January 2018) that Boissonnat, Nielsen, and Nock make the
following claim: Banerjee et al have shown that there is a bijection
between the regular exponential families in statistics and a subset of
the Bregman divergences called regular Bregman divergences (285).

\section{Jeffrey M. Lee: Manifolds and Differential Geometry}
\label{lee}

\subsection{The Tangent Structure (Chapter 2)}
\label{subsection:woochong}

\subsubsection{A Multivariable Calculus Exercise}
\label{subsubsection:edohmien}

Here is a nice exercise for my multivariable calculus students.
Consider the vector

\begin{equation}
  \label{eq:egaiyeih}
  \vec{OP}=(0.2,0.3,\sqrt{0.87})^{\intercal}
\end{equation}

$P$ is on the unit sphere. Provide the equation of the tangent plane.
There are two ways to do this (2179):

\begin{itemize}
\item use partial derivatives $m_{x},m_{y}$ and the plane equation
  $z=z_{0}+m_{x}(x-x_{0})+m_{y}(y-y_{0})$
\item use the dot product and the fact that $\vec{OP}$ is orthogonal
  to the tangent plane
\end{itemize}

\subsubsection{Lemma 2.4}
\label{subsubsection:fiezothu}

Showing $(f\circ\gamma_{v})'(0)=(f\circ{}c)'(0)$ on page 59 was a bit
of a challenge. Interestingly, understanding this is important for
solving exercise 2.16 on page 67 later on. On (2178) I am still
carrying on with the assumption on Lee, page 56, that this is the
special case of a submanifold of $\mathbb{R}^{N}$, but given Lee's
talk about the ``general setting'' on the bottom of page 57, this must
be incorrect. Show that

\begin{equation}
  \label{eq:chaequah}
  \left.\frac{d}{dt}\right\vert_{t=0}(f\circ{}x_{\alpha}^{-1}\circ{}h)(t)=D(f\circ{}x_{\alpha}^{-1})(x_{\alpha}(p))\cdot\vec{v}
\end{equation}

with $x_{\alpha}(p)=\vec{q}\in\mathbb{R}^{n}$, $p\in\mathcal{M}$, and
$h(t)=x_{\alpha}(p)+t\cdot\vec{v}$, $\vec{v}\in\mathbb{R}^{n}$. $f$ is
a smooth real-valued function defined on $\mathcal{M}$, which has a
chart $(U_{\alpha},x_{\alpha})$. Use the chain rule,

\begin{equation}
  \label{eq:baijoofe}
  D(f\circ{}x_{\alpha}^{-1}\circ{}h)(0)=D(f\circ{}x_{\alpha}^{-1})(h(0))D(h)(0)=D(f\circ{}x_{\alpha}^{-1})(x_{\alpha}(p))\cdot\vec{v}
\end{equation}

and also

\begin{equation}
  \label{eq:aongivoo}
  (f\circ{}c)'(0)=(f\circ{}x_{\alpha}^{-1}\circ{}x_{\alpha}\circ{}c)'(0)=D(f\circ{}x_{\alpha}^{-1})(x_{\alpha}(p))(x_{\alpha}\circ{}c)'(0)
\end{equation}

(see (2207)). 

\subsubsection{The Leibniz Law}
\label{subsubsection:eelahbao}

For the use of the Leibniz Law on page 61 note that
$C^{\infty}(\mathcal{M})$ is the set of real-valued, smooth functions
on $\mathcal{M}$. Think of tangent spaces as a straightening of the
manifold at a point. $T_{p}\mathcal{M}$ (kin) is the straightforward
geometric interpretation, $T_{p}\mathcal{M}$ (phys) uses
transformation laws, and $T_{p}\mathcal{M}$ (alg) uses linear
functions which fulfill the Leibniz Law. Either way it's conversion
therapy for manifolds: straighten them out at a point $p$ by using the
linear functions provided by differentiation.

\subsubsection{Theorem 2.10}
\label{subsubsection:eengoroo}

There is a proof on page 63 in Lee that 

\begin{equation}
  \label{eq:eisaanou}
  \left(\left.\frac{\partial}{\partial{}x^{1}}\right\vert_{p},\ldots,\left.\frac{\partial}{\partial{}x^{n}}\right\vert_{p}\right)
\end{equation}

is a basis for $T_{p}\mathcal{M}$ (alg). First, note that $g$ needs to
be defined all along $u$ from $0$ to $u$, thus the remark about the
convex set. Next, let's have a look at the claim that

\begin{equation}
  \label{eq:seenithe}
  g=g(0)+\sum{}g_{i}u^{i}
\end{equation}

based on the FTC. Let $g:x(U)\rightarrow\mathbb{R}$ be smooth. Let
$h:\mathbb{R}\rightarrow{}x(U)$ be defined by $h(t)=t\vec{u}$. Then

\begin{equation}
  \label{eq:avaiteiz}
  h'(t)=\left(
    \begin{array}{c}
      u^{1} \\
\vdots \\
u^{n}
    \end{array}\right)^{\intercal}
\end{equation}

and 

\begin{equation}
  \label{eq:jeeyongu}
  (g\circ{}h)'(t)=g'(h(t))\cdot{}h'(t)
\end{equation}

using matrix multiplication. Thus

\begin{equation}
  \label{eq:gaichieb}
  (g\circ{}h)'(t)=\sum_{i=1}^{n}\frac{\partial{}g}{\partial{}u^{i}}(h(t))u^{i}
\end{equation}

and by the FTC

\begin{equation}
  \label{eq:yioqueep}
  \int_{0}^{1}(g\circ{}h)'(t)\,dt=g(u)-g(0).
\end{equation}

It follows that

\begin{equation}
  \label{eq:eetuwuph}
  \int_{0}^{1}(g\circ{}h)'(t)dt=\int_{0}^{1}\sum_{i=1}^{n}\frac{\partial{}g}{\partial{}u^{i}}(h(t))u^{i}\,dt=\notag
\end{equation}

\begin{equation}
  \label{eq:ahwoofom}
\sum_{i=1}^{n}\left(\int_{0}^{1}\frac{\partial{}g}{\partial{}u^{i}}(t\vec{u})\,dt\right)u^{i}=\sum_{i=1}^{n}g_{i}(u)u^{i}=g(u)-g(0).
\end{equation}

\subsubsection{Exercise 2.16}
\label{subsection:aexaiwah}

Show that $T_{p}f(\beta{}w_{p})=\beta{}T_{p}f(w_{p})$. Let
$[c]=w_{p}$. Let $(U_{\alpha},x_{\alpha})$ be an arbitrary chart of
$\mathcal{M}$ and $(\tilde{U}_{\alpha},\tilde{x}_{\alpha})$ an
arbitrary chart of $\mathcal{N}$. By definition (see Lee, page 58f),

\begin{equation}
  \label{eq:aiquoowo}
  T_{p}f(\beta{}w_{p})=[f\circ\gamma_{1}]\mbox{ and }\beta{}T_{p}f(w_{p})=[\gamma_{2}]
\end{equation}

with $\gamma_{1}(t)=(x_{\alpha}^{-1}\circ\phi_{1})(t)$ and
$\gamma_{2}(t)=(\tilde{x}_{\alpha}^{-1}\circ\phi_{2})(t)$.

\begin{equation}
  \label{eq:engainok}
  \phi_{1}(t)=x_{\alpha}(p)+t\beta(x_{\alpha}\circ{}c)'(0)
\end{equation}

\begin{equation}
  \label{eq:pheequai}
  \phi_{2}(t)=\tilde{x}_{\alpha}(q)+t\beta(\tilde{x}_{\alpha}\circ{}f\circ{}c)'(0)
\end{equation}

(see box A on (2201) and box B on (2203)). Show that for all
$g\in{}C^{\infty}(\mathcal{N})$

\begin{equation}
  \label{eq:haithead}
  (g\circ{}f\circ\gamma_{1})'(0)=(g\circ\gamma_{2})'(0)
\end{equation}

Use the chain rule to separate $\phi_{1}$ and $\phi_{2}$, so the claim
reduces to LHS=RHS for

\begin{equation}
  \label{eq:ohngooze}
  \mbox{LHS}=D(g\circ{}f\circ{}x_{\alpha}^{-1})(x_{\alpha}(p))\circ{}D(\phi_{1})(0)
\end{equation}

\begin{equation}
  \label{eq:uboowaef}
  \mbox{RHS}=D(g\circ{}\tilde{x}_{\alpha}^{-1})(x_{\alpha}(f(p)))\circ{}D(\phi_{2})(0)
\end{equation}

with

\begin{equation}
  \label{eq:jahgeito}
D(\phi_{1})(0)=\beta{}D(x_{\alpha}\circ{}c)(0)  
\end{equation}

and

\begin{equation}
  \label{eq:ohquutee}
D(\phi_{2})(0)=\beta{}D(\tilde{x}_{\alpha}\circ{}f\circ{}c)(0).
\end{equation}

Now reconstitute (undo the chain rule) for

\begin{equation}
  \label{eq:zeicaero}
\mbox{LHS}=\beta{}D(g\circ{}f\circ{}x_{\alpha}^{-1}\circ{}x_{\alpha}\circ{}c)(0)  
\end{equation}

\begin{equation}
  \label{eq:xaimieha}
\mbox{RHS}=\beta{}D(g\circ{}\tilde{x}_{\alpha}^{-1}\circ{}\tilde{x}_{\alpha}\circ{}f\circ{}c)(0)  
\end{equation}

which are obviously equal to each other (see (2211)). To show that
$T_{p}f(v_{p}+w_{p})=T_{p}f(v_{p})+T_{p}f(w_{p})$ use the same
procedure. Let $v_{p}=[c_{v}]$ and $w_{p}=[c_{w}]$. Then you need to
show that

\begin{equation}
  \label{eq:xubeasey}
  (g\circ{}f\circ\gamma_{1})'(0)=(g\circ\gamma_{2})'(0)
\end{equation}

where

\begin{equation}
  \label{eq:ahngokee}
  \gamma_{1}(t)=x_{\alpha}^{-1}(x_{\alpha}(p)+t((x_{\alpha}\circ{}c_{v})'(0)+(x_{\alpha}\circ{}c_{w})'(0)))
\end{equation}

and

\begin{equation}
  \label{eq:yiateegu}
  \gamma_{2}(t)=\tilde{x}_{\alpha}^{-1}(\tilde{x}_{\alpha}(f(p))+t((\tilde{x}_{\alpha}\circ{}f\circ{}c_{v})'(0)+(\tilde{x}_{\alpha}\circ{}f\circ{}c_{w})'(0)))
\end{equation}

Use the distributive law of matrix multiplication to show
(\ref{eq:xubeasey}) (see (2213)). Now show that $T_{p}f$ is a linear
isomorphism if $f$ is a diffeomorphism. For surjectivity, let
$[\hat{c}]\in{}T_{q}\mathcal{N}$. Define
$c:t\rightarrow(f^{-1}\circ\hat{c})(t)$. Then $[f\circ{}c]=[\hat{c}]$.
For injectivity, let $c_{1},c_{2}$ be such that
$[f\circ{}c_{1}]=[f\circ{}c_{2}]$ and let
$g:\mathcal{M}\rightarrow\mathbb{R}$ be smooth. Then $g\circ{}f^{-1}$
is a smooth function from $\mathcal{N}\rightarrow\mathbb{R}$ and
therefore 

\begin{equation}
  \label{eq:chahshau}
  (g\circ{}c_{1})'(0)=(g\circ{}f^{-1}\circ{}f\circ{}c_{1})'(0)=(g\circ{}f^{-1}\circ{}f\circ{}c_{2})'(0)=(g\circ{}c_{2})'(0)
\end{equation}

which establishes $[c_{1}]=[c_{2}]$ (for this proof see (2215)).\hspace\fill $\square$

\subsubsection{Definition 2.18}
\label{subsubsection:yiobeche}

On page 68, Lee claims that the definition of
$T_{p}\mathcal{M}(v_{p})$ is independent of the representative of
$v_{p}$. Let $(p,v_{1},(U,x_{1})),(p,v_{2},(U,x_{2}))$ be two
representatives of $v_{p}$. Let $(V,y)$ be an arbitrary chart for
$\mathcal{N}$ (I suppose we should also show independence of this
chart for $\mathcal{N}$). Then $T_{p}\mathcal{M}(v_{p})$ is defined to
be represented by $(q,w,(V,y))$ with

\begin{equation}
  \label{eq:aexapief}
  w=\left.{}D(y\circ{}f\circ{}x^{-1})\right\vert_{x(p)}\cdot{}v_{p}
\end{equation}

(this is a correction of an omission on Lee's part). The two
representatives corresponding to the two charts for $\mathcal{M}$ are
$(q,w_{1},(V,y))$ and $(q,w_{2},(V,y))$ with 

\begin{equation}
  \label{eq:oucootoo}
w_{1}=D(y\circ{}f\circ{}x_{1}^{-1})\cdot{}v_{1}
\end{equation}

\begin{equation}
  \label{eq:ieneipae}
w_{2}=D(y\circ{}f\circ{}x_{2}^{-1})\cdot{}v_{2}
\end{equation}

We need to show that

\begin{equation}
  \label{eq:chuaphoo}
  w_{2}=\left.D(y\circ{}y^{-1})\right\vert_{y(q)}\cdot{}w_{1}.
\end{equation}

You show this by taking
$\left.{}D(y\circ{}y^{-1}\circ{}y\circ{}f\circ{}x_{1}^{-1}\circ{}x_{1}\circ{}x_{2}^{-1})\right\vert_{x_{2}(p)}\cdot{}v_{2}$
and using (\ref{eq:oucootoo}) and (\ref{eq:ieneipae}) on the one hand,
the chain rule on the other hand (see (2216f)).

\subsubsection{Definition 2.20}
\label{subsubsection:queceixo}

A word about the definition of differentials on page 69. Using
derivations, a tangent vector $v_{p}$ is a linear map from
$C^{\infty}(\mathcal{M})$ into the real numbers obeying Leibniz's Law.
The space of these linear maps is $T_{p}\mathcal{M}$. Consider a
function $f\in{}C^{\infty}(\mathcal{M})$. Now define a linear map $df$
assigning to each $v_{p}$ the real number $v_{p}\cdot{}f$. $df$ is a
function from $T_{p}\mathcal{M}$ to the real numbers and depends on
$f$ and $p$. 

Consider the dual space (the cotangent space) of $T_{p}\mathcal{M}$.
It is defined as the vector space spanned by
$(v_{1}^{\ast},\ldots,v_{n}^{\ast})$, where $(v_{1},\ldots,v_{n})$ is
a basis for $T_{p}\mathcal{M}$ and $v_{i}^{\ast}$ is a function
from $T_{p}\mathcal{M}$ to the real numbers with
$v_{i}^{\ast}(v_{j})=\delta_{ij}$. Lee's claim is that
$df\in{}T_{p}^{\ast}\mathcal{M}$. 

Now look at this in basic calculus. The differential $dy$ is defined
as a function assigning to each $dx$ the real number $f'(x)dx$. Think
of it as a linear function from the tangent space of $f$ at point $a$
into the real numbers. $dx$ is usually identified with $\Delta{}x$,
but now you can see the subtle difference between $\Delta{}x$ and
$dx$. If $f:\mathbb{R}^{2}\rightarrow\mathbb{R}$, then $dx$ is a
vector in $\mathbb{R}^{2}$ and gets mapped to 

\begin{equation}
  \label{eq:kaithaed}
  dy=df(p)(dx)=\frac{\partial{}f}{\partial{}x_{1}}dx_{1}+\frac{\partial{}f}{\partial{}x_{2}}dx_{2}
\end{equation}

where $dx_{1}$ is the $x$-component of $dx$ and $dx_{2}$ is the
$y$-component. The $z$-component of $dx$ is $dy$ by definition (see
the diagram on (2223)). For (\ref{eq:kaithaed}), see (2222). The
upshot here is that $dx$ is a vector in $T_{p}\mathcal{M}$, while
$\Delta{}x$ is a vector in the $xy$-plane, and we can generalize

\begin{equation}
  \label{eq:lajiobie}
  dy=f'(x)dx
\end{equation}

to

\begin{equation}
  \label{eq:oopeodoo}
  dy=df(p)(dx)=\sum_{i=1}^{n}\frac{\partial{}f}{\partial{}x_{i}}dx_{i}
\end{equation}

where the last expression is only defined for
$f:\mathbb{R}^{n}\rightarrow\mathbb{R}$. Now let's see if Lee is
correct with his claim that $df\in{}T_{p}^{\ast}\mathcal{M}$ and that
the generalization in (\ref{eq:oopeodoo}) holds.

We have a vector space $T_{p}\mathcal{M}$. Therefore, there is a dual
vector space $T_{p}^{\ast}\mathcal{M}$. We know (theorem 2.10) that
$T_{p}\mathcal{M}$ has the basis

\begin{equation}
  \label{eq:feiwawee}
    \left(\left.\frac{\partial}{\partial{}x^{1}}\right\vert_{p},\ldots,\left.\frac{\partial}{\partial{}x^{n}}\right\vert_{p}\right)
\end{equation}

The basis for $T_{p}^{\ast}\mathcal{M}$ is
$(v_{1}^{\ast},\ldots,v_{n}^{\ast})$ such that 

\begin{equation}
  \label{eq:bieyangi}
  v_{i}^{\ast}\left(\left.\frac{\partial}{\partial{}x^{j}}\right\vert_{p}\right)=\delta^{ij}
\end{equation}

If $(U,x)$ is a chart, then the differentials of the coordinate
functions $x^{1},\ldots,x^{n}$ fulfill (\ref{eq:bieyangi}) because

\begin{equation}
  \label{eq:yooxieta}
  \left.dx^{i}\right\vert_{p}\left(\left.\frac{\partial}{\partial{}x^{j}}\right\vert_{p}\right)=\frac{\partial{}x^{i}}{\partial{}x^{j}}(p)=\delta^{ij}
\end{equation}

From now on, I will follow Lee and write $M$ and $N$ for manifolds,
rather than $\mathcal{M}$ and $\mathcal{N}$.

\subsubsection{Another Remark on Definition 2.21}
\label{subsubsection:doopifae}

Let $T_{p}f$ be a tangent map from
$T_{p}M\rightarrow{}T_{f(p)}\mathbb{R}$ and $df(p)$ be a differential
of $f$ at $p$. Lee's claim is that $T_{p}f$ and $df(p)$ basically do
the same thing once you identify $T_{f(p)}\mathbb{R}$ with
$\mathbb{R}$ using the natural isomorphism on page 66. Therefore, a
differential is just a special tangent map into $T_{f(p)}\mathbb{R}$
(identified with $\mathbb{R}$). 

To see why this is so note that
$T_{p}f:T_{p}M\rightarrow{}T_{f(p)}\mathbb{R}$ is the map that sends
$v_{p}$ to $T_{p}f(v_{p})$, a linear map which sends any real-valued
function $g$ to the real number
\begin{equation}
  \label{eq:ugaiwoon}
  (T_{p}f(v_{p}))(g)=v_{p}(g\circ{}f)
\end{equation}

Note also that $df(p)$ is a real-valued map sending $v_{p}$ to
$v_{p}f$. This means we have succeeded once $T_{p}f(v_{p})$ is
identified with $v_{p}f$ via the natural isomorphism $\jmath$. Consider the
real number $v_{p}f$. 
\begin{equation}
  \label{eq:aezuroti}
  \jmath(v_{p}f)=[\hat{c}]
\end{equation}
where
\begin{equation}
  \label{eq:aecoocha}
  \hat{c}(t)=f(p)+t\cdot(v_{p}f)
\end{equation}
Once we show that $[\hat{c}]=T_{p}f(v_{p})$ we are done. Note that
$[\hat{c}]\in{}T_{f(p)}\mathbb{R}$.

Let $g$ be a real-valued function in $C^{\infty}(M)$ and $[c]=v_{p}$.
Use the interpretation on page 65, the chain rule, and
(\ref{eq:aecoocha}) to show that
\begin{equation}
  \label{eq:iephavam}
(T_{p}f(v_{p}))(g)=v_{p}(g\circ{}f)=\left.\frac{d}{dt}\right\vert_{t=0}(g\circ{}f\circ{}c)=\notag
\end{equation}
\begin{equation}
  \label{eq:hekeivei}
g'(f(p))f'(p)c'(0)=g'(f(p))(v_{p}f)=\left.\frac{d}{dt}\right\vert_{t=0}(g\circ{}\hat{c})
\end{equation}

\subsubsection{Definition 2.21}
\label{subsubsection:ipaikaph}

Consider Lee's comment on the bottom of page 69. Let $\gamma$ be a smooth
curve from a real neighbourhood of $0$ to $[a,b]=I$ such that 

\begin{equation}
  \label{eq:wohyeisu}
  [\gamma]=\left.\frac{\partial}{\partial{}u}\right\vert_{t_{0}}
\end{equation}

According to the interpretation of page 65 (identifying kinematic and
algebraic tangent spaces),

\begin{equation}
  \label{eq:oengueko}
\left.\frac{d}{dt}\right\vert_{t=0}f\circ\gamma=\left(\left.\frac{\partial}{\partial{}u}\right\vert_{t_{0}}\right)(f)
\end{equation}

Let $f=\mbox{id}$ for

\begin{equation}
  \label{eq:aexoosho}
\left.\frac{d}{dt}\right\vert_{t=0}\gamma=\left.\frac{\partial\mbox{id}}{\partial{}u}\right\vert_{t_{0}}=1
\end{equation}

Now use the same interpretation in reverse for

\begin{equation}
  \label{eq:cebaasee}
  \dot{c}(t_{0})\cdot{}f=\left(T_{t_{0}}c\left(\left.\frac{\partial}{\partial{}u}\right\vert_{t_{0}}\right)\right)(f)=\left.\frac{d}{dt}\right\vert_{t=0}f\circ(c\circ\gamma)=\notag
\end{equation}

\begin{equation}
  \label{eq:cebaasee}
  \left.\frac{d}{dt}\right\vert_{t=t_{0}}f\circ{}c\cdot\left.\frac{d}{dt}\right\vert_{t=0}\gamma=\left.\frac{d}{dt}\right\vert_{t=t_{0}}f\circ{}c
\end{equation}

as claimed by Lee. This is a nice example how a mixed use of algebraic
and kinematic tangent spaces can be reconciled using the
interpretations of page 65. I suppose an alternative way of doing this
is to use the algebraic definition of a tangent map $T_{p}f$ instead
(definition 2.19).

\subsubsection{Exercise 2.23}
\label{subsubsection:jaighaec}

$T_{p}f=0$ for all $p\in{}M$ means that for any smooth
$g:N\rightarrow{}\mathbb{R}$ and for any $v_{p}=[c]$ it is true that

\begin{equation}
  \label{eq:thuiriaf}
  v_{p}(g\circ{}f)=0
\end{equation}

Let $p_{1}$ and $p_{2}$ be two arbitrary connected points in $M$ such
that a curve $c:I\rightarrow{}M$ connects them with
$c(a)=p_{1},c(0)=p,c(b)=p_{2},a<0<b$. Then (\ref{eq:thuiriaf}) is true
for $v_{p}$ and therefore

\begin{equation}
  \label{eq:yeiphaes}
  v_{p}(g\circ{}f)=\left.\frac{d}{dt}\right\vert_{t=0}\left(g\circ{}f\circ{}c\right)=0
\end{equation}

Define $h(t)=(g\circ{}f\circ{}c)(t)$ for an arbitrary $g$. Claim:
$h'(t)=0$ for all $t$ in the open interval $(a,b)$. Assume that
$h'(t_{0})\neq{}0$. Then define

\begin{equation}
  \label{eq:ohngeihi}
  \phi(t)=r_{2}t^{2}+r_{1}t+t_{0}
\end{equation}

with

\begin{equation}
  \label{eq:laucaequ}
  r_{1}=\frac{b^{2}(a-t_{0})-a^{2}(b-t_{0})}{ab(b-a)}
\end{equation}

\begin{equation}
  \label{eq:ohjireeg}
  r_{2}=\frac{a(b-t_{0})-b(a-t_{0})}{ab(b-a)}.
\end{equation}

Then $\phi(a)=a,\phi(0)=t_{0},\phi(b)=b$. Also,

\begin{equation}
  \label{eq:aifoheif}
  \phi'(0)=r_{1}\neq{}0
\end{equation}

because $b^{2}(a-t_{0})\neq{}a^{2}(b-t_{0})$ ($a-t_{0}$ is the only
negative term). Let $\hat{c}=c\circ\phi$ and

\begin{equation}
  \label{eq:xeixaiyo}
  (g\circ{}f\circ\hat{c})'(0)=0
\end{equation}

for the same reason as (\ref{eq:yeiphaes}). However

\begin{equation}
  \label{eq:aingooda}
  (g\circ{}f\circ\hat{c})'(0)=(g\circ{}f\circ{}c\circ\phi)'(0)=(g\circ{}f\circ{}c)'(\phi(0))\cdot\phi'(0)
\end{equation}

and by assumption and (\ref{eq:aifoheif}) both of these factors are
not equal to zero. The contradiction gives us the claim that $h'(t)=0$
for all $t\in{}(a,b)$. According to the Fundamental Theorem of
Calculus, there is a $\hat{s}$ with $a\leq\hat{s}\leq{}b$ giving us
(together with our claim)

\begin{equation}
  \label{eq:eitheisu}
  h(b)-h(a)=h'(\hat{s})(b-a)=0
\end{equation}

Therefore $(g\circ{}f)(p_{1})=(g\circ{}f)(p_{2})$ for arbitrary $g$.
This is only possible if $f(p_{1})=f(p_{2})$. Therefore, $f$ is
locally constant (2229--2237).\hspace\fill $\square$

\subsubsection{Theorem 2.25}
\label{subsubsection:ooghohbe}

Lee claims that under the assumptions of Theorem 2.25 $M$ and $N$ have
the same dimension. Remember that 

\begin{equation}
  \label{eq:naijahba}
  \left(\left.\frac{\partial}{\partial{}x^{1}}\right\vert_{p},\ldots,\left.\frac{\partial}{\partial{}x^{n}}\right\vert_{p}\right)
\end{equation}

is a basis for $T_{p}M$. Therefore, $\mbox{dim}(M)=\mbox{dim}(T_{p}M)$
and $\mbox{dim}(N)=\mbox{dim}(T_{q}N)$. If there is a linear
isomorphism between two vector spaces $V$ and $W$, then
$\mbox{dim}(V)=\mbox{dim}(W)$. For a proof by S.F. Ellermeyer, see

\begin{alltt}
\footnotesize
  http://ksuweb.kennesaw.edu/~sellerme/sfehtml/classes/math3260/isomorphicvectorspaces.pdf
\end{alltt}

Since $T_{p}f$ is such a linear isomorphism between $T_{p}M$ and
$T_{q}N$, it follows that $\mbox{dim}(M)=\mbox{dim}(N)$.\hspace\fill
$\square$

Notice how Exercise 2.16 and Theorem 2.25 are related. I needed
Exercise 2.16 for the claim that $T_{p}f$ is a \emph{linear}
isomorphism. $D(y\circ{}f\circ{}x^{-1})(x(p))$ is a linear isomorphism
because it is one of the definitions (Definition 2.18) of a tangent
map $T_{p}f$, and $T_{p}f$ is a linear isomorphism by assumption. If
$y\circ{}f\circ{}x^{-1}$ is injective, then $f$ is also injective. Let
$\pi_{i}=x^{-1}(p_{i})$ and $f(p_{1})=f(p_{2})$. Then
$\pi_{1}=\pi_{2}$ and therefore $p_{1}=p_{2}$. Restricting the
codomain of $f$ to $f(O)$ means that $f|_{O}$ is a diffeomorphism on
$O$.

\subsubsection{Lemma 2.28 Partials Lemma}
\label{subsubsection:geesohla}

My work is on (2329--2332). $(v,w)$ is prima facie not a tangent
vector. It becomes one by identification with
$(T\imath^{q}+T\imath_{p})(v,w)$, see Lee, page 73. $T\imath^{q}$ is not the
inverse of $T_{(p,q)}\mbox{pr}_{1}$, as the diagram near the middle of
page 73 in Lee suggests. It is the tangent map with respect to the
insertion function $\imath^{q}$. Insertion $\imath^{q}$ and stripping
$T_{(p,q)}\mbox{pr}_{1}$ are not inverse to each other. The validity
of the lemma is now easily demonstrated.

\begin{equation}
  \label{eq:uwiusosa}
  (T_{(p,q)}f(v,w))(g)\stackrel{(1)}{=}(v,w)(g\circ{}f)\stackrel{(2)}{=}v(g\circ{}f\circ\imath^{q})+w(g\circ{}f\circ\imath_{p})\stackrel{(3)}{=}\notag
\end{equation}
\begin{equation}
  \label{eq:gaedaiqu}
  (\partial_{1}f_{(p,q)}v)(g)+(\partial_{2}f_{(p,q)}w)(g)
\end{equation}

(1) is true based on the definition of tangent maps (definition 2.19
on page 68) with respect to $f$. (2) is true based on the
identification referred to above and the definition, again, of tangent
maps, this time with respect to $\imath^{q}$ and $\imath_{p}$. (3) is
true based on the definition of partial tangent maps on page 72.\hspace\fill $\square$

\subsubsection{Definition 2.30}
\label{subsubsection:iehaifab}

I needed help from math stackexchange to understand why a countable union of
zero measure sets, under the given description, has measure zero. See
Anthony Peter's answer to questin 1156907 in math stackexchange. The
idea is to pick
\begin{equation}
  \label{eq:ooghoiph}
\epsilon_{i}=\frac{\epsilon}{2^{i}}
\end{equation}
for
\begin{equation}
  \label{eq:fiaweeba}
  A=\bigcup_{i\in\mathbb{N}}A_{i}.
\end{equation}
Then the sum of the cube volumes will not exceed $\epsilon$, even
though there are countably many of these sums.

\subsubsection{Morse Lemma 2.5.1}
\label{subsubsection:haixifie}

Lee makes the claim that $df|_{p_{e}}=T_{p_{e}}f=0$. Definition 2.20
  tells us that $df|_{p_{e}}$ is a function
  $T_{p_{e}}M\rightarrow\mathbb{R}$ which maps $v_{p_{e}}$ to
  $v_{p_{e}}f$. To make sense of Lee's claim, $v_{p_{e}}f$ (which is
  in $\mathbb{R}$) needs to be identified with $u_{f(p_{e})}$ in
  $T_{f(p_{e})}\mathbb{R}$. We use the identification procedure on
  page 66. Let $g\in{}C^{\infty}U$ defined on a neighbourhood
  $U\subset\mathbb{R}$ containing $f(p_{e})$. Then 
  \begin{equation}
    \label{eq:zooxohgh}
    u_{f(p_{e})}(g)=\left.\frac{d}{dt}\right\vert_{t=0}g(f(p_{e})+t(v_{p_{e}}f))=g'(f(p_{e}))(v_{p_{e}}f)
  \end{equation}
Note that
\begin{equation}
  \label{eq:gaivahra}
  g'(f(p_{e}))(v_{p_{e}}f)=\left.\frac{d}{dt}\right\vert_{t=0}g\circ{}\gamma
\end{equation}
where $[\gamma]=u_{f(p_{e})}$ and (\ref{eq:gaivahra}) is true because
of the kinematic interpretation we can give to the algebraic version
of a tangent vector (see page 65). Since $T_{p_{e}}f$ sends
$v_{p_{e}}$ to $[f\circ{}c]$, where $v_{p_{e}}=[c]$, $T_{p_{e}}f$ and
$df|_{p_{e}}$ are equal via the identification
$\mathbb{R}\leftrightarrow{}T_{f(p_{e})}\mathbb{R}$ if
\begin{equation}
  \label{eq:aishaiza}
  (g\circ\gamma)'(0)=(g\circ{}f\circ{}c)'(0)
\end{equation}
for all smooth $g:U\rightarrow\mathbb{R}$. (\ref{eq:aishaiza}) is
equivalent to
\begin{equation}
  \label{eq:iegahnon}
g'(f(p_{e}))(v_{p_{e}}f)=g'(f(p_{e}))(f\circ{}c)'(0)
\end{equation}
which is true if $v_{p_{e}}f=(f\circ{}c)'(0)$, but that is just the
formula for setting up the interpretation between the algebraic and
kinematic version of the tangent space $T_{p_{e}}M$ (see page 65). Now
we need to show that $T_{p_{e}}f=0$ (again via the identification
$\mathbb{R}\leftrightarrow{}T_{f(p_{e})}\mathbb{R}$). $0$ corresponds
to $[\hat{\gamma}]$ with
\begin{equation}
  \label{eq:quietoph}
  \hat{\gamma}(t)=f(p_{e})+t\cdot{}0=f(p_{e})
\end{equation}
a constant function for which $\hat{\gamma}'(0)=0$. The claim
$T_{p_{e}}f=0$ is true if and only if
\begin{equation}
  \label{eq:aimeceij}
  (g\circ{}f\circ{}c)'(0)=(g\circ\hat{\gamma})'(0)=0
\end{equation}
Since we also have
\begin{equation}
  \label{eq:quauluog}
  (g\circ{}f\circ{}c)'(0)=g'(f(p_{e}))\cdot(f\circ{}c)'(0)
\end{equation}
by the chain rule, (\ref{eq:aimeceij}) is true if
\begin{equation}
  \label{eq:ohnienga}
  (f\circ{}c)'(0)=0
\end{equation}
This holds because $f\circ{}c$ has an extremum (oBdA a maximum) at $0$
and
\begin{equation}
  \label{eq:fiofooci}
  0\leq\lim_{h\rightarrow{}0^{-}}\frac{(f\circ{}c)(h)-(f\circ{}c)(0)}{h}=\lim_{h\rightarrow{}0}\frac{(f\circ{}c)(h)-(f\circ{}c)(0)}{h}=\notag
\end{equation}
\begin{equation}
  \label{eq:shohcath}
  (f\circ{}c)'(0)=\lim_{h\rightarrow{}0^{+}}\frac{(f\circ{}c)(h)-(f\circ{}c)(0)}{h}\leq{}0
\end{equation}

\subsubsection{Exercise 2.37}
\label{subsubsection:aijeiloj}

This really needs elucidation from Problem 13 (which, by the way, is
at the end of chapter 2 in the exercises section) and section 2.8,
vector fields. Lie derivatives should be very helpful in showing
$X_{p}(Yf)=Y_{p}(Xf)$.

The first problem is to show that the Hessian $[H_{f,p}]_{x}$ is
invertible if and only if $[H_{f,p}]_{\hat{x}}$ is invertible, where
$\hat{x}$ is an alternative coordinate system on $M$. I needed to
learn a lot about the multivariable second-derivative test and the
associated linear algebra before I could tackle this question.
Williamson and Trotter in \emph{Multivariable Mathematics} give a nice
overview of why the second derivative provides conditions for extrema,
especially

\begin{equation}
  \label{eq:iefuetho}
  g(x)=g(0)+g'(0)x+\int_{0}^{x}(x-t)g''(t)\,dt
\end{equation}

for $g(x)=f(x_{0}+xu)$, where $u$ is an arbitrary unit vector. Use
partial integration to see that (\ref{eq:iefuetho}) is true;
substitute $f$ in (\ref{eq:iefuetho}) to see that $f$ has a minimum
where $\partial^{2}f/\partial{}u^{2}$ is positive. (It is unclear to
me how Theorem 4.4 in WT relates to Theorem 4.5 and 4.6, this would be
worth pursuing.)

The linear algebra bit is in R.J. Goult, \emph{Applied Linear
  Algebra}, a nondescript little brown book I have on my bookshelf.
Section 3.5 about quadratic forms provides the connection between
multivariable second derivatives and matrices, sections 5.3 and 5.4
provide background with respect to eigenvalues and eigenvectors of
real symmetric matrices, which Hessians are.

Let $f:M\rightarrow{}\mathbb{R}$ and $x,y$ be two coordinate functions
for a neighbourhood of $p_{e}\in{}M$. $p_{e}$ is a critical point for
$f$, so $df$ at $p_{e}$ is zero. Now let the Hessian
$[H_{f,p_{e}}]_{x}$ be singular. Show that $[H_{f,p_{e}}]_{y}$ is also
singular. This establishes that nondegeneracy is well-defined
according to the first definition (definition 2.36).

To show this, I am interested in $f\circ{}y^{-1}$ and whether it is
degenerate, given that $f\circ{}x^{-1}$ is degenerate. Since
\begin{equation}
  \label{eq:quaiquuo}
f\circ{}y^{-1}=f\circ{}x^{-1}\circ{}x\circ{}y^{-1}
\end{equation}
the question reduces to the following: given degenerate
$g=f\circ{}x^{-1}$ and differentiable $z=x\circ{}y^{-1}$ show that
$g\circ{}z$ is degenerate. Note that
$g:\mathbb{R}^{n}\rightarrow\mathbb{R}$ and
$z:\mathbb{R}^{n}\rightarrow\mathbb{R}^{n}$.

The reduction only works if $g=f\circ{}x^{-1}$ is degenerate. I will
show this now. Let $x(p_{e})=t_{0}$. Define
$c_{i}(t)=x^{-1}\circ{}b_{i}$ with $b_{i}(t)=t_{0}+t\cdot{}u_{i}$,
where $b_{i}:\mathbb{R}\rightarrow\mathbb{R}^{n}$ and $u_{i}$ is the
usual $i$-th basis vector of $\mathbb{R}^{n}$. Because
$c_{i}(0)=p_{e}$ there exists a $v_{p_{e}}^{(i)}$ such that
$[c_{i}]=v_{p_{e}}^{(i)}$. On the one hand, for fixed but arbitrary $i$,
\begin{equation}
  \label{eq:uvoiquie}
   \left.\frac{d}{dt}\right\vert_{t=0}(f\circ{}c_{i})=v_{p_{e}}^{(i)}\cdot{}f=df(p_{e})\cdot{}v_{p_{e}}^{(i)}=0
\end{equation}
because $p_{e}$ is a critical point. On the other hand
\begin{equation}
  \label{eq:fushahch}
    \left.\frac{d}{dt}\right\vert_{t=0}(f\circ{}c_{i})=\left.\frac{d}{dt}\right\vert_{t=0}(f\circ{}x^{-1}\circ{}b_{i})=\left.\nabla{}f\circ{}x^{-1}\right\vert_{b_{i}(0)}\cdot{}b_{i}'(0)=\left(\nabla(f\circ{}x^{-1})(t_{0})\right)\cdot{}u_{i}
\end{equation}
This means that the $i$-th component of
$\nabla(f\circ{}x^{-1})(t_{0})$ is zero, but since $i$ was arbitrary,
$t_{0}$ is a critical point of $f\circ{}x^{-1}$. 

\bibliographystyle{ChicagoReedweb} 
\bibliography{bib-2902}

\end{document}
